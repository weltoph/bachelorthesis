\section{Concrete Semantics}
	Because \emph{\acp{HG}} are used to represent the current state of the heap
	the semantics of the programming language are modelled in
	terms of graph transformations. Therefore the different statements of a
	program are connected with changes of the shape of the hypergraph
	representing the state of the heap. In the following the necessary graph
	transformations which are based on presented transformations of \cite{fmsd}
	for modelling the semantics are presented. This is done in two separate
	parts where firstly structural changes are discussed and secondly
	transformations that regard the \emph{\acp{PE}} of the graph.
\subsection{Graph Transformations}
\label{sec:graphtrans}
	The structural graph transformations are presented in the following:
	\begin{itemize}
		\item $H[+v]$ adds a new fresh node $v$ to $H$, which is of universally
			type, thus there are edges attached to for all \emph{selectors}
		\item $H[\setminus E']$ removes the set of edges $E'$ from $H$
		\item $H[x\hookrightarrow v]$ connects one $x$ labeled edge with $v$,
			where $x\in\Var$
		\item $H[u\xhookrightarrow{s} v]$ connects one $s$ labeled edge
			from the node $u$ to the node $v$
		\item $H[+_{\rho}n\rightrightarrows v_1\cdots v_m]$ adds a fresh $n$
			labeled edge with \emph{permission} $\rho$ connected to
			$v_1\cdots v_m$
	\end{itemize}
	Formally these transformations are defined as follows, where $v$ is a new
	node and $e,e_{1},\dots$ are new edges:
	\begin{center}
		\begin{tabular}{lcl}
			$H[+v]$ & $=$ &
				$\begin{aligned}
					(&V_{H}\uplus\{v\},
					E_{H}\uplus\{e_{1},\dots,e_{|\mathit{Sel}|}\},\\
					&\con_{H}\cup\{e_{1}\mapsto vv_{\text{null}},\dots,
					e_{|\mathit{Sel}|}\mapsto vv_{\text{null}}\},\\
					&\lab_{H}\cup\{e_1\mapsto\mathit{enum}_{\mathit{Sel}}(1),\dots,
					e_{|\mathit{Sel}|}\mapsto\mathit{enum}_{\mathit{Sel}}
					(|\mathit{Sel}|)\},\\
					&\ext_{H},
					\perm_{H}\cup\{e_{1}\mapsto\mathit{WR},\dots,
					e_{|\mathit{Sel}|}\mapsto\mathit{WR}\})\\
				\end{aligned}$\\
			\hline
			$H[\setminus E']$ & $=$ &
				$\begin{aligned}
					(&V_H,E_H\setminus E',
					 \con_H\upharpoonright (E\setminus E'),
					 \lab_H\upharpoonright (E\setminus E'),\\
					 &\ext_H,
					 \perm_H\upharpoonright (E\setminus E'))
				\end{aligned}$\\
			\hline
				$H[x\hookrightarrow u]$ & $=$ &
				\begin{minipage}{4.5cm}
					$\begin{aligned}
						(&V_{H},E_{H},\\
						 &\mathit{con}_{H}[e\mapsto u],\\
						 &\mathit{lab}_{H},\mathit{ext}_{H},\mathit{perm}_{H})
					\end{aligned}$
				\end{minipage}%
				\begin{minipage}{5cm}
					for one $e$ with $\lab_{H}(e) = x$, if such an $e$ exists
				\end{minipage}\\
			\hline
				$H[u\xhookrightarrow{s} u']$ & $=$ &
				\begin{minipage}{4.5cm}
					$\begin{aligned}
						(&V_H,E_H,\\
						 &\mathit{con}_H[e\mapsto uu'],\\
						 &\mathit{lab}_H,\mathit{ext}_H,\mathit{perm}_H)
					\end{aligned}$
				\end{minipage}%
				\begin{minipage}{5cm}
					if there is $e$ with $\lab_{H}(e) = s$ and $\con_{H}(e)(1) = u$
					and $u,u'\in V_{H}$
				\end{minipage}
			\\
			\hline
			$H[+_{\rho}n\rightrightarrows v_1\cdots v_n]$ & $=$ &
			$\begin{aligned}
				(&V_H,E_H\uplus\{e\},
				 \con_H\cup\{e\mapsto v_1\cdots v_n\},\\
				 &\lab_H\cup\{e\mapsto n\},
				 \ext_H,
				 \perm_H\cup\{e\mapsto\rho\})
			\end{aligned}$\\
		\end{tabular}
	\end{center}
	The few more transformation which focus on various aspects of the
	\emph{permissions} are also introduced below:
	\begin{itemize}
		\item $H[\downarrow t]$ denotes that $t\in\PI$
			does not longer identify a certain process (for example by reassigning
			$t$)
		\item $H[\leftarrow T]$ denotes that the \emph{permissions} for
			$T\in\mathbb{P}(\PI)$ are returned
		\item $H[t = t']$ is used to describe that for
			$t,t'\in\PI$ $t$ now identifies the same
			process as $t'$
		\item $H[E - T]$ denotes that for the edges in $E$ the token $T$ is added
			to the $\mathit{PermSet}$ of the \emph{permissions}
	\end{itemize}
	The following definitions formalise these intuitions:
	\begin{definition}[Dropped Thread Variable]
		For an \emph{\ac{HC}} $H$, $t\in\PI$ let
		$E_{\text{remove}}\subseteq E_{H}$ be all edges $e$ such that
		$\lab_{H}(e) = N_{\{t\}}$. Then $H[{\downarrow{t}}]$ denotes the \ac{HC}
		with:
		\begin{equation*}
			H[{\downarrow{t}}]= (V_{H},
			\underbrace{E_{H}\setminus E_{\text{remove}}}_{=: E'},
			\con_{H} \upharpoonright E',
			\lab' \upharpoonright E', \perm'
			\upharpoonright E')
		\end{equation*}
		where $\lab'$ mirrors $\lab_{H}$ except of all edges $e$ such that
		$\lab_{H} = N_{T}\in\mathbb{T}$ where holds
		$\lab'(e) = N_{(T\setminus \{t\})}$. And for all edges $e$ with
		$\perm_{H}(e) = \rho_{e} - \Phi_{e}$ $\perm'(e)$ is defined as follows:
		\begin{equation*}
			\perm'(e) =
			\begin{cases}
				\mathit{WR^\ast} - \Phi_{e}\setminus\{\{t\}\}&\text{if }
				\rho_{e}\in\{\mathit{WR},\mathit{WR^\ast}\} \text{ and }
				\{t\}\in\Phi_{e}\\
				\mathit{RD^\ast} - \Phi_{e}\setminus\{\{t\}\}&\text{if }
				\rho_{e}\in\{\mathit{RD},\mathit{RD^\ast}\} \text{ and }
				\{t\}\in\Phi_{e}\\
				\rho_{e} - \{T\setminus\{t\}\mid T\in\Phi_{e}\} &\text{otherwise}
			\end{cases}
		\end{equation*}
	\end{definition}
	Note that all edges labeled with $N_{\{t\}}$ are removed. This is because
	these edges are the placeholder for the part of the heap that the process
	$t$ referred to might alter. Thus, in order to ensure data race freedom this
	part of the heap has to be exclusively accessible by this process. Because
	the process which operates on this part of the heap can never be rejoined
	(the token is a singleton set only containing $t$, thus only $t$ refers to
	this process and is reassigned) all these nodes and edges
	must never be accessed. To achieve this the placeholder is removed which
	removes this part of the heap permanently. Furthermore all tokens containing
	$t$ are updated in order to indicate that $t$ no longer identifies the same
	process as the other process identifier of this token. For those
	$\mathit{PermSets}$ that contain the token $\{t\}$ the $\mathit{BasePerm}$
	is \enquote{starred} in order to indicate that, since there is no further
	reference to the process the corresponding access token can never be
	returned. And this implies that the $\mathit{BasePerm}$ can never be fully
	recovered.
	\begin{definition}[Returned Token]
		For $H\in\HC$, $T\in\mathbb{P}(\PI)$
		\begin{equation*}
			H[\leftarrow T] = (V_{H}, E_{H}, \con_{H}, \lab_{H}, \ext_{H}, \perm')
		\end{equation*}
		with $\perm'(e) = \rho - \Phi\setminus\{T\}$ where
		$\perm_{H}(e) = \rho -\Phi$.
	\end{definition}
	Note that simply the token $T$ is removed from all the $\mathit{PermSets}$.
	The return of parts of the heap that the process identified by $T$ (or
	rather all $t\in T$) had exclusive access on is dealt with separately.
	\begin{definition}[Thread Variable Assignment]
		For $H\in\HC$ and $t,t'\in\PI$ let
		$\Phi_{t'} = \{ T\in\Phi \mid t'\in T\}$ denote the set of all
		tokens in a $\mathit{PermSet}$ $\Phi$ which contain the process
		identifier $t'$. Then
		\begin{equation*}
			H[t = t'] = (V_{H[\downarrow t]}, E_{H[\downarrow t]},
				\con_{H[\downarrow t]}, \lab',
				\ext_{H[\downarrow t]}, \perm')
		\end{equation*}
		where $\lab'$ mirrors $\lab_{H[\downarrow t]}$ except for all
		$e\in E_{H[\downarrow t]}$ with $\lab_{H[\downarrow t]}(e) = N_{T}$ and
		$t'\in T$ where $\lab'(e) = N_{T\cup\{t\}}$. And
		$\perm'(e)=\rho-(\Phi\setminus\Phi_{t'}\cup
		\{T\cup\{t\}\mid T\in\Phi_{ t'}\})$ if
		$\perm_{H[\downarrow t]}(e) = \rho-\Phi$.
	\end{definition}
	Note that for the assignment the reassigned identifier is previously
	dropped since its original value will be lost after the assignment. And
	the set of tokens for all $\mathit{PermSets}$ that do not contain the
	process identifier $t'$ are simply preserved. To thoses tokens that contain
	$t'$ as process identifier $t$ is added since $t$ identifies from now on the
	same process as $t'$.
	\begin{definition}[Add Token to Edge]
		For $H\in\HC$, $T\in\mathbb{P}(\PI)$ and $E\subseteq E_{H}$
		\begin{equation*}
			H[E-T] = (V_{H}, E_{H}, \con_{H}, \lab_{H}, \ext_{H}, \perm')
		\end{equation*}
		where
		\begin{equation*}
			\perm'(e) =
			\begin{cases}
				\perm_{H}(e) - T &\text{if }e\in E\\
				\perm_{H}(e)     &\text{otherwise}\\
			\end{cases}
		\end{equation*}
	\end{definition}
	The token $T$ is simply added to all $\mathit{PermSets}$ of edges in $E$.
	\subsection{Semantics}
	In the following the semantics of programs are defined
	by graph transformations. Therefore some relations over nodes in
	\emph{\acp{HG}} are defined as follows:
	\begin{itemize}
		\item $v\xhookrightarrow{s}_{\rho}v'$ states that the node $v$ is
			connected to the node $v'$ by an edge labeled with $s\in\mathit{Sel}$
			and the permission $\rho$
		\item $x\hookrightarrow_{\rho}v$ denotes that there is an edge labeled
			with $x\in\mathit{Var}$ and the permission $\rho$ which is connected
			to the node $v$
	\end{itemize}
	For both relations the permission might be omitted to indicate that one
	permission $\rho$ exists such that the relation holds.

	At first the semantics of pointer expressions are examined.
	This is, evaluating a pointer expression under an \emph{\ac{HC}}
	$H\in\mathit{HC}_\Sigma$ ($\llbracket\cdot\rrbracket_H$). Intuitively, this
	means that variables are identified with the node the edge labeled with the
	variable is attached to, the \textbf{null} value is identified with
	$v_{\text{null}}$ and dereferencing a selector of the object corresponds to
	following the outgoing edge from a node labeled by this selector. There are
	two possible errors: firstly, dereferencing anything from the \textbf{null}
	value and secondly, accessing a selector which is not there. The second case
	indicates that a selector is accessed although another process demands
	exclusive access. Because of the universal type of every node the only
	possible way a selector is absent is because it is hidden behind a
	placeholder for another process or removed because the reference to the
	corresponding process is lost. Such an error is indicated by returning an
	error symbol $\bot$. Formally this leads to the following:
	\begin{center}
		\begin{tabular}{lcl}
			$\llbracket \textbf{null}\rrbracket_H$ & $=$ & $v_{\text{null}}$\\
			&&\\
				$\llbracket x\rrbracket_{H}$ & $=$ & $v$\hspace{0.65cm} with
				$x\hookrightarrow v$\\
			&&\\
			$\llbracket x.s\rrbracket$ & $=$ & $
			\begin{cases}
				v    &\text{if } \llbracket x\rrbracket_{H}\xhookrightarrow{s} v\\
				\bot &\text{otherwise}\\
			\end{cases}
			$\\
		\end{tabular}
	\end{center}
	Note that for an \emph{\ac{HC}} these semantics are well defined since
	every node has maximal one outgoing edge for every selector and for every
	variable exists exactly one edge labeled with it.

	The semantics of conditions are evaluated very intuitively, but where $\bot$
	propagates strictly, i.e. if there is one expression yielding $\bot$ the
	whole evaluation becomes $\bot$. Therefore for pointer expressions
	$p_1,p_2$, and conditions $c_1, c_2$:
	\begin{center}
		\begin{tabular}{lcl}
			$\llbracket p_1 == p_2\rrbracket_H$ & $=$ & $
			\begin{cases}
				\bot &\text{if } \llbracket p_1\rrbracket_H = \bot \text{ or }
					\llbracket p_2\rrbracket_H = \bot \\
				\mathit{true} &\text{if } \llbracket p_1\rrbracket_H =\llbracket
					p_2\rrbracket_H\\
				\mathit{false} &\text{if } \llbracket p_1\rrbracket_H
					\neq\llbracket p_2\rrbracket_H \\
			\end{cases}$\\
			\\
			$\llbracket p_1\;\text{!=}\; p_2\rrbracket_H$ & $=$ & $
			\begin{cases}
				\bot &\text{if } \llbracket p_1\rrbracket_H = \bot \text{ or }
					\llbracket p_2\rrbracket_H = \bot \\
				\mathit{false} &\text{if } \llbracket p_1\rrbracket_H=\llbracket
					p_2\rrbracket_H\\
				\mathit{true} &\text{if } \llbracket p_1\rrbracket_H
					\neq\llbracket p_2\rrbracket_H \\
			\end{cases}$\\
			\\
			$\llbracket c_1\;\&\&\; c_2\rrbracket_H$ & $=$ &
				$\begin{cases}
					\bot &\text{if } \llbracket c_1\rrbracket_H = \bot \text{ or }
					\llbracket c_2\rrbracket_H = \bot\\
					\llbracket c_1\rrbracket_H\land\llbracket c_2\rrbracket_H &
					\text{otherwise}\\
				\end{cases}
					$\\
			\\
			$\llbracket c_1 \;\|\; c_2\rrbracket_H$ & $=$ &
				$\begin{cases}
					\bot&\text{if }\llbracket c_1\rrbracket_H =\bot\text{ or }
				\llbracket c_2\rrbracket_H=\bot\\
					\llbracket c_1\rrbracket_H\lor \llbracket c_2\rrbracket_H
					&\text{otherwise}\\
				\end{cases}$\\
		\end{tabular}
	\end{center}
	This gives the basis to define the semantics for the given programming
	language using a \emph{transition system}:
	\begin{definition}[Semantics of the Programming Language]
		The \emph{semantics} of the individual statements are modelled by a
		\emph{transition relation} of the form
		$\rhd \subseteq
		(\mathbb{S}\cup\{\varepsilon\} \times \mathit{HC}_{\Sigma})^{2}$ where
		$\mathbb{S}$ denotes the language derived by the grammar in
		section \ref{sec:syntax}.
	\end{definition}
	The semantics for the \emph{assignment, allocation, loop, conditional
	and skip statement} can be given by the following rules, where
	$v\hookrightarrow_{\rho} -$ abbreviates
	$\exists v'.v\hookrightarrow_{\rho} v'$ and $v\xhookrightarrow{s}_{\rho} -$
	is expanded to $\exists v'.v\xhookrightarrow{s}_{\rho} v'$:
	\begin{center}
		\label{transitionrules}
		\begin{prooftree}
			\AxiomC{$\llbracket P\rrbracket_H\neq\bot$}
			\AxiomC{$\llbracket x\rrbracket_H\hookrightarrow_{\mathit{WR}} -$}
			\BinaryInfC{$(x = P, H)\rhd (\varepsilon, H[
				\llbracket x\rrbracket_H\hookrightarrow
				\llbracket P\rrbracket_H])$}
		\end{prooftree}
		\begin{prooftree}
			\AxiomC{$\llbracket P\rrbracket_H\neq\bot$}
			\AxiomC{$\llbracket x\rrbracket_H\xhookrightarrow{s}_{\mathit{WR}} -$}
			\BinaryInfC{$(x.s = P, H)\rhd (\varepsilon, H[
			\llbracket x\rrbracket_H\xhookrightarrow{s}
		\llbracket P\rrbracket_H])$}
		\end{prooftree}
		\begin{prooftree}
			\AxiomC{$\llbracket C\rrbracket_H = \text{true}$}
			\UnaryInfC{$(\text{while}(C) \text{ do } S \text{ done}, H)
			\rhd (S;\text{while}(C) \text{ do } S \text{ done}, H)$}
		\end{prooftree}
		\begin{prooftree}
			\AxiomC{$\llbracket C\rrbracket_H = \text{false}$}
			\UnaryInfC{$(\text{while}(C) \text{ do } S \text{ done}, H)
			\rhd (\varepsilon, H)$}
		\end{prooftree}
		\begin{prooftree}
			\AxiomC{$\llbracket C\rrbracket_H = \text{true}$}
			\UnaryInfC{$(\text{if}(C) \text{ then } S_1
			\text{ else } S_2 \text{ fi}, H)\rhd(S_1,H)$}
		\end{prooftree}
		\begin{prooftree}
			\AxiomC{$\llbracket C\rrbracket_H = \text{false}$}
			\UnaryInfC{$(\text{if}(C) \text{ then } S_1
			\text{ else } S_2 \text{ fi}, H)\rhd(S_2,H)$}
		\end{prooftree}
		\begin{prooftree}
			\AxiomC{$\llbracket x\rrbracket_H \hookrightarrow_{\mathit{WR}} -$}
			\UnaryInfC{$(\text{new}(x), H)\rhd(\varepsilon,H[+v]
			[x\hookrightarrow v])$}
		\end{prooftree}
		\begin{prooftree}
			\AxiomC{}
			\UnaryInfC{$(\text{skip}, H)\rhd(\varepsilon,H)$}
		\end{prooftree}
		\begin{prooftree}
			\AxiomC{$(S_1,H)\rhd(S_{1}', H')$}
			\UnaryInfC{$(S_1;S_2, H)\rhd(S_{1}';S_2,H')$}
		\end{prooftree}
		\begin{prooftree}
			\AxiomC{$t,t'\in\PI$}
			\UnaryInfC{$(t = t', H)\rhd(\varepsilon, H[t = t'])$}
		\end{prooftree}
		\begin{prooftree}
			\AxiomC{}
			\UnaryInfC{$(\varepsilon;S, H)\rhd(S,H)$}
		\end{prooftree}
	\end{center}
	Note that for operations that change an edge a $\mathit{WR}$ permission on
	that edge is expected. Furthermore every statement that does not have a
	fitting transition rule yields explicitly $\bot$.
	Since fork and join are more complex they are examined in more detail below.

	\subsection{Fork}
	\label{sec:fork}
	As described previously fork creates a process and provides it with
	parameters.This is similar to procedure calls modelled in
	\cite{ProcedureSummaries}.
	In the following two types of parameters are distinguished:
	\begin{itemize}
		\item \emph{Formal parameters}, which belong to the newly created process
		\item \emph{Actual parameters}, which belong to the process executing the
			fork statement
	\end{itemize}
	In the call of a fork statement the actual parameters are provided to
	the newly created process and are identified with formal parameters.
	Thus the new process initially only knows the nodes identified by the actual
	parameters and can now access other nodes via the different selectors.
	The subgraph that can be accessed in this way is called
	\emph{reachable}. The concept of \emph{reachability} will be later on
	introduced formally but partially relies on the abstraction that is
	introduced in Section \ref{sec:abstraction} and is therefore here only given
	as the set $\reach(v_1,\dots,v_n)$ which denotes all edges that are
	considered to be reachable from the nodes $v_1,\dots,v_n$.\footnote{actually
		the set $\reach(v_1,\dots,v_n)$ is a superset of the actual reachable
		edges, but because it covers at least all actual reachable edges the
	soundness of the model for the semantics is ensured}

	Reconsider the example from Listing \ref{lst:ExampleProgram} and Figure
	\ref{fig:BinTreeProgLang} on page \pageref{fig:BinTreeProgLang} where it was
	observed that \emph{forking} a new process and assigning it to a process
	identifier potentially causes to lose the possibility to refer to the
	process the process identifier identified before. In order to model this
	loss of information the semantics rely on a corresponding graph
	transformation ($H[\downarrow t]$). Another aspect of the fork
	statement is that as demonstrated in Figure
	\ref{fig:BinTreeProgLangPermReplacement} it is possible that a write access
	ticket is completely transferred to the \emph{forked} process. As mentioned
	in Section \ref{sec:introduction} the behaviour of processes and especially
	which parts of the heap the process might alter is defined in a
	\emph{contract}.
	These contracts consist of a \emph{precondition}, the set of edges which the
	process might alter and a \emph{postcondition}.
	The precondition is the representation of the context from which the process
	is forked from. The postcondition is the representation of the
	\emph{heap} that resulted from the execution of the process started from an
	initial heap state.
	This initial heap state is obtained by a \emph{transformation} of the
	precondition which is examined later on.
	Firstly, as mentioned in Section \ref{sec:ProgLang} it is expected
	that every process has its own variables as well as process identifier.
	W. l. o. g. the heap state for every forked process can be described as an
	\emph{\ac{HC}} over the set $\PI'$ and
	$\Sigma' = \Sel\uplus\mathbb{T}'\uplus\Var'$ with renaming of variables and
	process identifier as well as limiting the used variables and process
	identifier to subsets of $\Var'$ and $\PI'$. Especially it holds that
	$\Var\cap\Var'=\emptyset$ and $\PI\cap\PI'=\emptyset$.
	Secondly, the precondition is examined as a representation which is isomorph
	to the reachable subgraph. Therefore it is assumed that the precondition is
	a \emph{\ac{HG}} from $\HG$. On the other hand the postcondition is the
	result of the computation of the forked process and therefore represented as
	\emph{\ac{HG}} from $\mathit{HG}_{\Sigma'}^{\PI'}$. And this computation of
	the forked process ensures that only edges from the alternable set are
	altered. These coherences are illustrated graphical in Figure
	\ref{fig:forkHeaps}.
	\begin{figure}[ht]
		\begin{center}
			\begin{tikzpicture}[node distance=3cm]
	%states
	\node [textnode] (P) {Precondition};
	\node [textnode, below=of P] (R) {Reachable Subgraph};
	\node [textnode, right=of P] (I) {Initial Heap State};
	\node [textnode] (Q) at (R-|I)   {Postcondition};
	%cong
	\node [below=1.5cm and 1.5cm of P] {$\cong$};

	\begin{scope}[on background layer]
		\node [fill=yellow!50, fit=(P) (R), label={90:forking process ($\HG$):}] {};
		\node [fill=green!30,  fit=(I) (Q), label={90:forked  process ($\mathit{HG}^{\PI'}_{\Sigma'}$):}] {};
	\end{scope}

	%computation
	\draw [very thick,->, decorate,
	decoration={snake,amplitude=.7mm, segment length=4mm, post length=2mm}]
	(I) to node[left, text width=1.8cm]{Execution} node[right, text width=2cm]{(with respect to alternable set)}(Q);
	\draw [very thick,->, decorate,
	decoration={snake,amplitude=.7mm, segment length=4mm, post length=2mm}]
	(P) to node[above]{Transformation} (I);

	
\end{tikzpicture}

		\end{center}
		\caption{Graphical representation of the coherences between precondition,
		initial heap state, reachable subgraph and postcondition}
		\label{fig:forkHeaps}
	\end{figure}
	For every program the set of contracts is formally described as a function
	\begin{equation*}
		\Cont\colon\Proc\rightarrow
		\mathbb{P}(\HG\times\mathbb{P}(E)\times\mathit{HG}_{\Sigma'}^{\PI'})
	\end{equation*}
	where for every program $m$ and every contract $C=(P_C, E_C, Q_C)\in
	\Cont(m)$ holds that $E\subseteq E^{P_C}_{\mathit{WR}}$, thus the set of
	alterable edges ($E_{C}$) is a subset of the edges with write access of the
	precondition. 

	The necessary graph transformations associated with the fork
	statement are examined in the following. Let $H$ be an \emph{\ac{HG}} and
	$C = (P_{C}, E, Q_{C})\in\mathit{Cont}(m)$. Let $\ap_{1},\dots,\ap_{n}$ be
	the nodes that are identified by the actual parameter of the fork statement.
	This means that for the statement
	\begin{lstlisting}
t = fork(m($x_{1},\dots,x_{n}$));
	\end{lstlisting}
	with the current heap representation $H$ it holds that
	$\ap_{i} = \llbracket x_{i}\rrbracket_{H}$ for all $1\leq i\leq n$.
	Thus the set of reachable edges from the actual parameter is defined as
	$\reach(\ap_{1},\dots,\ap_{n})$.
	The actual precondition for the newly forked process is
	the current \emph{heap} representation reduced to the subgraph which is
	induced by the reachable part. This subgraph is defined as follows:
	\begin{definition}[Edge Induced Subgraph]
		Let $H$ be a  \emph{\ac{HG}} and $E'\subseteq E_{H}$ a set of edges. Then
		the \emph{subgraph induced by $E'$} is defined as
		\begin{equation*}
			H\upharpoonright E' = (\bigcup_{e\in E'}\Lbag\con_{H}(e)\Rbag, E',
			\lab\upharpoonright E', \con\upharpoonright E',\ext\upharpoonright E',
			\perm\upharpoonright E')
		\end{equation*}
	\end{definition}
	To find the fitting contract $C$ for the fork statement it is necessary that
	the reachable subgraph forms the precondition of $C$ and the actual
	parameters agree with the formal parameters. Let $\fp_{1},\dots,\fp_{n}$
	denote the nodes of the formal parameters within the precondition of $C$.
	Then the reachable subgraph is structurally identical to the precondition
	and the actual parameters agree with the formal parameters. Let $R$ denote
	the reachable subgraph of a fork statement and $P$ denote the fitting
	precondition. Then $R$ and $P$ are isomorphic and especially this
	isomorphism matches formal and actual parameter. Let $R\afiso P$
	denote the existence of such an isomorphism.

	Once again reconsider the example in Figure \ref{fig:BinTreeProgLangPerm}
	and let
	\begin{equation*}
		C = \left(\parbox{5cm}{\resizebox{5cm}{!}{\input{tikz/precondition}}},
		\{e_2,e_3,e_4, e_7, e_8\},
		\parbox{5cm}{\resizebox{5cm}{!}{\input{tikz/postcondition}}}\right)
	\end{equation*}
	be a possible contract such that $C\in\Cont(\mathit{traverse})$ and $v_1$ is
	the only formal parameter. Here $l$ and $r$ abbreviate $\mathit{left}$ and
	$\mathit{right}$ respectively. Note that $C$ fulfills that the alterable set
	is a subset of the edges with $\mathit{WR}$ permission of the precondition.
	Secondly, in the postcondition only edges have been altered which were part
	of the alterable set (namely $e_3, e_4$). Thirdly, forking by $C$ leads to
	the replacement seen in Figure \ref{fig:BinTreeProgLangPermReplacement}
	because the induced subgraph of $\{e_2,e_3,e_4,e_7,e_8\}$ is replaced. As it
	can be seen in Figure \ref{fig:BinTreeProgLangPermReplacement} the
	replacement is connected to the node that is identified with $v_1$ and
	$v_{\text{null}}$.  This is because $v_1$ represents an object which is part
	of the main process as well as the newly forked one. Such nodes are called
	\emph{border} nodes and are defined as those nodes that are connected to
	edges of the alterable set and to edges that are not part of the alterable
	set. Additionally, the placeholder in Figure
	\ref{fig:BinTreeProgLangPermReplacement} is also
	connected to $v_{\text{null}}$, which is generally assumed for all
	placeholder, thus the heap representation of every process can refer to the
	same distinguishable \textbf{null} value. The formal definition of the set
	of border nodes is closely connected to the definition
	of reachability and is therefore likewise postponed to Section
	\ref{sec:abstractsemantics}. For now the set of
	border nodes for a contract $C$ and a heap representation $H$ is denoted by
	$\border_H(C)$\footnote{just like the reachable set this set is potentially
	a superset of the actual border nodes}. These border nodes represent the
	connection points between the heap representation of the main process and
	the part of the heap of the forked process that can be altered. For joining
	the process later on it is crucial to identify these nodes in the
	postcondition. This is achieved by using the mechanism of \emph{external
	nodes} as introduced for \emph{hyperedge replacement} in Section
	\ref{sec:hyperedgereplacement}.
	With these considerations in mind the transformation of the heap
	representation that models the fork statement can be described.
	Let therefore
	\begin{itemize}
		\item $t$ = \textbf{fork}($m(x_1,\dots,x_n)$);
		\item with $H$ as the current heap representation,
		\item $\ap_{i} = \llbracket x_{i}\rrbracket_{H}$ for all $1\leq i\leq n$,
		\item $(P_{C},E_{C},Q_{C})\in\Cont(m)$,
		\item $P_{C}\afiso (H\upharpoonright \reach(\ap_{1},\dots,\ap_{n}))$,
		\item $t\in\PI$,
		\item $b_{C}(H)$ denote the set of border nodes,
		\item $\enum_{b_{C}(H)} =
			\enum_{b_{C}(H)}(1)\dots\enum_{b_{C}(H)}(|b_{C}(H)|)$ is an arbitrary
			sequence of the border nodes
	\end{itemize}
	W. l. o. g. it is assumed that $P_C$ shares nodes and edges with $H$ which
	can easily achieved by renaming the edges and nodes of $P_C$ according to
	the existing isomorphism. Then the graph transformation can be given as
	follows:
	\begin{equation*}
		\label{eq:H'}
		H' = \underbrace{\overbrace{\underbrace{
		\overbrace{H[\downarrow t]}^{\circled{1}}[\setminus E_{C}]}_{\circled{2}}
		[+_{\mathit{WR}}N_{\{t\}}\rightrightarrows
		\enum_{b_{C}(H)}]}^{\circled{3}}[(E_{P}\setminus E_{C})-\{t\}]}_{
		\circled{4}}
	\end{equation*}
	where the different steps state the following:
	\begin{enumerate}[label=\protect\circled{\arabic*}]
		\item the process identifier used for the newly forked process is freed
		\item the edges that the forked process demands write access on are
			removed from the heap representation
		\item the formerly removed edges are replaced with an hyperedge
		\item to all edges that can be read by the newly forked process the
			appropiate token is added
	\end{enumerate}
	Formally this process is therefore described by the following transition
	rule:
	\begin{prooftree}
		\label{prooftree:fork}
		\AxiomC{$(P_{C}, E_{C}, Q_{C})\in\mathit{Cont}(m)$}
		\AxiomC{$P_{C}\afiso (H\upharpoonright \reach
		(\llbracket p_{1}\rrbracket_{H},\dots,\llbracket p_{n}\rrbracket_{H}))$}
		\BinaryInfC{$(t = \text{fork}(m(p_1,\dots,p_n)), H)\rhd
			(\varepsilon, H')$}
	\end{prooftree}

	For the \emph{join} statement some formalisms of \emph{\aclp*{HRG}}, namely
	\emph{\acl*{HR}} is used and therefore the definition is postponed and
	\acl*{HR} is introduced before.

\subsection{Hyperedge Replacement}
\label{sec:hyperedgereplacement}
	In the following the concept of \emph{\ac{HR}} is presented.
	\emph{\Ac{HR}} is a  graph transformations which replaces single hyperedges
	by more complex graphs \cite[p. 104]{HandbookGraphGrammars}. As the fork
	statement replaces parts of the heap by a single hyperedge in order to
	\enquote{hide} this subgraph, rendering it unavailable for the main process.
	Indeed the \enquote{hidden} subgraph still exists within the heap
	representation of the forked process and can especially be returned by a
	\emph{join} statement. Therefore a way to re-integrate this subgraph is
	discussed in the following. A possible way to present rules for substituting
	hyperedges by hypergraphs are \emph{production rules} which have the form
	$p\colon X\rightarrow H$. Here $X$ is a label and $H$ a hypergraph and this
	production rule $p$ can be applied to a hypergraph by removing a
	$X$-labeled edge and replacing it by the hypergraph $H$. For better
	understanding the production rule in Figure \ref{fig:productionrule} is
	considered in the following. On the right hand side of this production rule
	the nodes marked by $1$ and $2$ are \emph{external nodes} of the hypergraph
	and their annotation is their position in the sequence $\ext$. For the
	hyperedges that are labeled with $B$ the numbers denote the position of the
	nodes in the sequence of connected nodes.
	\begin{figure}[ht!]
		\begin{center}
			\begin{tikzpicture}
	\node (p) {$p:$};
	\node[he, right=0.1cm of p] (lhs) {$B$};
	\node[node, right=3.5cm of lhs] (v2) {};
	\node[node, above right=of v2, ext] (v1) {1};
	\node[node, below right=of v1] (v3) {};
	\node[he, below=of v2] (e1) {$B$};
	\node[he, below=of v3] (e2) {$B$};
	\node[node, left=of e1, ext] (v4) {2};

	\draw[very thick, ->, shorten >=2.2cm, shorten <=7pt] (lhs) to (v2);
	\draw[sel] (v1) to node[left ]{$l$} (v2);
	\draw[sel] (v1) to node[right]{$r$} (v3);
	\draw[connector] (e1) to node[right]{1} (v2);
	\draw[connector] (e1) to node[above]{2} (v4);
	\draw[connector, bend left] (e2) to node[above]{2} (v4);
	\draw[connector] (e2) to node[right]{1} (v3);
\end{tikzpicture}

			\caption{A possible production (permissions are omitted)}
			\label{fig:productionrule}
		\end{center}
	\end{figure}
	In order to replace for instance such a $B$-labeled hyperedge in an
	\emph{\ac{HG}} the external nodes of the right hand side of the production
	rule are identified with the nodes connected to the hyperedge. A simple
	example (see Figure \ref{fig:replacement}) illustrates such a replacement.
	Note that the external nodes and the connected nodes that are identified
	had the same position within the respective sequences.
	\begin{figure}
		\begin{center}
			\begin{tikzpicture}
	%graph
	\node[node] (s1) {};
	\node[node, below right=of s1] (s3) {};
	\node[node, node distance=2 and 2, below left =of s3] (s2) {$n$};
	\node[he, below=of s3] (e3) {$B$};

	\draw[sel] (s1) to node[left ]{$l$} (s2);
	\draw[sel] (s1) to node[right]{$r$} (s3);
	\draw[connector] (e3) to node[above]{2} (s2);
	\draw[connector] (e3) to node[right]{1} (s3);

	%rhs of production rule:
	\node[node, below=of e3, ext] (v1) {1};
	\node[node, below left=of v1] (v2) {};
	\node[node, below right=of v1] (v3) {};
	\node[he, below=of v2] (e1) {$B$};
	\node[he, below=of v3] (e2) {$B$};
	\node[node, left=of e1, ext] (v4) {2};

	\draw[sel] (v1) to node[left ]{$l$} (v2);
	\draw[sel] (v1) to node[right]{$r$} (v3);
	\draw[connector] (e1) to node[right]{1} (v2);
	\draw[connector] (e1) to node[above]{2} (v4);
	\draw[connector, bend left] (e2) to node[above]{2} (v4);
	\draw[connector] (e2) to node[right]{1} (v3);

	%connection
	\draw[dashed, ->, bend right=45] (v1) to (s3);
	\draw[dashed, ->, bend left=45]  (v4) to (s2);

	%leadsto symbol
	\node[above right=1.2cm and 0.2cm of v3] {$\Rightarrow$};

	%result
	\node[node] (null) [right=of v3] {$n$};
	\node[he] (e1) [right=of null] {$B$};
	\node[node, above=of e1] (root-right-left) {};
	\node[node, above right=1cm and 0.5cm of root-right-left](root-right){};
	\node[node, below right=1cm and 0.5cm of root-right](root-right-right){};
	\node[node, above left=of root-right] (root) {};
	\node[he, below=of root-right-right] (e2) {$B$};

	\draw[sel] (root) to node[right]{$r$} (root-right);
	\draw[sel] (root) to node[left ]{$l$} (null);
	\draw[sel] (root-right) to node[left ]{$l$} (root-right-left);
	\draw[sel] (root-right) to node[right]{$r$} (root-right-right);
	\draw[connector] (e1) to node[left ]{1} (root-right-left);
	\draw[connector] (e2) to node[right]{1} (root-right-right);
	\draw[connector] (e1) to node[above]{2} (null);
	\draw[connector, bend left] (e2) to node[above]{2} (null);

\end{tikzpicture}

			\caption{Illustration of an \acl{HR}}
			\label{fig:replacement}
		\end{center}
	\end{figure}
	Such a replacement of an hyperedge is called a \emph{derivation} and is
	formally described as follows:
	\begin{definition}[Hyperedge Replacement]
		Let $p$ be a production rule of the form $p\colon X\rightarrow K$,
		$H\in\mathit{HG}_{\Sigma}$ a \emph{\ac{HG}} with an edge $e\in E_{H}$
		with $\mathit{lab}_{H}(e)=X$ and $|\ext_K| = |\con_{H}(e)|$. Then
		$H' = H[e/K]$ is an \emph{\ac{HG}} with:
		\begin{itemize}
			\item $V_{H'}=V_{H}\cup(V_{K}\setminus\Lbag \mathit{ext}_{K}\Rbag)$
			\item $E_{H'}=(E_{H}\setminus\{e\})\cup E_K$
			\item $\begin{aligned}
				\mathit{con}_{H'}=\mathit{con}_{H}\upharpoonright(E_{H} \setminus
				\{e\})\cup((\mathit{id}_{K}&[\ext_{K}(1)\mapsto\con_{H} (e)(1)]
				\dots\\
					&[\ext_{K}(|\ext_{K}|)\mapsto\con_{H}(e)(|\ext_{K}|)])\circ
				\mathit{con}_{K})
			\end{aligned}$
			\item $\mathit{lab}_{H'}=\mathit{lab}_{H}\upharpoonright(E_{H}
				\setminus\{e\})\cup\mathit{lab}_{K}$
			\item $\mathit{ext}_{H'}=\mathit{ext}_{H}$
			\item $\mathit{perm}_{H'}=\mathit{perm}_{H}\upharpoonright(E_{H}
				\setminus\{e\})\cup \mathit{perm}_{K}$
		\end{itemize}
	\end{definition}
	An \emph{\ac{HG}} $H'$ is derived from an \emph{\ac{HG}} $H$ by a production
	rule $p\colon X\rightarrow K$ (denoted by $H\xRightarrow{p}H'$) if $H'$ is
	isomorphic to $H[e/K]$ for any edge $e$ labeled with $X$. Furthermore
	$\Rightarrow^{\ast}$ is the transitive closure of this relation.
	\subsection{Join}
	\label{sec:join}
	Having introduced the mechanism of hyperedge replacement the postponed
	semantics of the join statement can now be addressed. Therefore
	it is distinguished between the \emph{joining} process which is the one
	actually executing the join statement and the \emph{joined} process
	which is the one that terminated and now returns its permissions.
	Returning this permissions can be done in two separate steps:
	Firstly, the permissions for the shared edges, i.e. those edges which
	$\mathit{BasePerm}$ is either $\mathit{RD}$ or $\mathit{RD^{\ast}}$ for the
	joined process, are returned.
	Since these edges are still present in the heap representation of the
	joining process returning those permissions is a
	matter of mutating the \emph{\ac{PE}} of the heap representation of
	the joining process. Secondly, the edges which were allowed to be
	altered are returned, i.e. edges which $\mathit{BasePerm}$ is either
	$\mathit{WR}$ or $\mathit{WR^{\ast}}$. This is done by replacing the
	hyperedge in the joining process which represents this part of the
	heap by its corresponding \emph{\ac{HG}} of the joined
	process. This imposes a further restriction to the contracts which have to
	ensure that the external nodes of the postcondition of the forked process
	are the same as the border nodes of the placeholder in the heap
	representation of the forking process. To do so, the sequence of external
	nodes in the precondition is used to encode which nodes are border nodes
	and the isomorphism $\afiso$ is restricted to map the border nodes in the
	heap representation of the forking process to the external nodes in the
	precondition. Following it is formally described how the initial heap state
	for the forked process is obtained from the contract
	$C = (P_C, E_C, Q_C)$:\footnote{which refers to the \emph{transformation} in
	Figure \ref{fig:forkHeaps}}
	Let $\ap_{1},\dots,\ap_{n}$ denote the nodes that mark the actual parameters
	and $\fp_{1},\dots,\fp_{1}$ denote the names of the formal paramters they
	are matched to (especially $\fp_{i}\in\Sigma'$ for all $1\leq i\leq n$).
	Furthermore let for all $s\in\Sel$ $s'\in\Sel'$ denote the corresponding
	selector used to describe the heap states of forked processes,
	then $I \in\mathit{HG}^{\PI'}_{\Sigma'}$ is the initial heap state for the
	forked process with:
	\begin{itemize}
		\item $V_I = \{v^{I}_{v'} \mid v'\in V_{P_{C}}\}$
		\item $E_I = \{e^{I}_{e'} \mid e'\in E_{P_{C}}\} \cup
			\{e_{\fp_i}\mid 1\leq i\leq n\}\cup\{e_{v}\mid v\in(\Var'\setminus
			\{\fp_{i}\mid 1\leq i\leq n\})\}$
		\item $\con_{I} = \{v'\mapsto v^{I}_{v'}\mid v'\in V_{P_{C}}\}
			\circ\con_{P_{C}}\cup \{e_{\fp_i}\mapsto\ap_i\mid 1\leq i\leq n\}
			\cup\{e_{v}\mapsto v_{\text{null}}\mid v\in(\Var'\setminus
			\{\fp_{i}\mid 1\leq i\leq n\})\}$
		\item $\lab_{I} = \{e_{\fp_{i}}\mapsto\fp_{i}\mid 1\leq i\leq n\}
			\cup\{e\mapsto s'\mid\lab_{P_{C}}(e) = s\}\cup\{e_{v}\mapsto v\mid
			v\in(\Var'\setminus \{\fp_{i}\mid 1\leq i\leq n\})\}$
		\item $\ext_I = \{v^{I}_{v'}\mapsto v'\mid v'\in V_{P_{C}}\}
			(\enum_{\border_{P_C}}(1)\dots\enum_{\border_{P_C}}(|\border_{P_{C}}
			(C)|))$
		\item $\perm_{I} = \{e\mapsto\mathit{WR}\mid e\in E_{C}\}\cup
			\{e\mapsto\mathit{RD}\mid e\in E_{P_{C}}\setminus E_{C}\}\cup
			\{e_{v}\mapsto\mathit{WR}\mid v\in(\Var'\setminus \{\fp_{i}\mid 1\leq
				i\leq n\})$
	\end{itemize}
	Note that actually only the selectors are renamed and variables introduced
	for the formal parameters that point to the nodes the actual parameters
	pointed to in the fork statement, also all other variables of $\Var'$ are
	introduced and initially point to $v_{\text{null}}$. Furthermore the
	external nodes of the
	initial heap state are used to mark the border nodes in order to ensure that
	the \enquote{$\mathit{WR}$-part} of the postcondition agrees on the shared
	objects when it is joined. At last, all permissions in the initial heap
	state are simple and grant those access tickets that the contract demands.
	From this initial heap state the execution of the forked program starts.

	For both steps of returning permissions for the join statement it is
	important to determine if all
	permissions are returned which were derived when the process was forked.
	This is not the case for all edges which \emph{\acp{PE}} are not
	\emph{simple} or which $\mathit{BasePerm}$ is \enquote{starred} (thus
	$\mathit{WR^\ast}$ or $\mathit{RD^\ast}$). These permissions are
	unrecoverably lost and this propagates strictly to the permissions of the
	heap representation of the joining process.
	
	For returning the read permissions it is avoided to match the
	edges from the heap representation of the joining and
	joined process since it imposes further difficulties later on.
	Therefore the minimal permission is returned: for all
	edges of the joining process from which permissions were 
	derived to the joined process the $\mathit{BasePerm}$ is 
	\enquote{starred} if there is \emph{any} edge in the postcondition
	for which the permission cannot be completely returned.

	Returning the write permissions is done by simply applying a
	production rule to the hyperedge of the joining process that
	represents the borrowed part. Let now
	$C = (P_{C}, E_{C}, Q_{C})\in\Cont(m)$ be the contract by which
	the joined process was forked (note that this contract has to be the same
	as for the fork statement that created the process which is re-joined now,
	in order to save the contract for matching it from fork to join statement
	it can be e. g. attached to the placeholder) and $T_t$ the set of all
	aliases of the process identifier $t$ that is used in the join statement.
	Further the following few definitions are used for convenience:
	\begin{equation*}
		\label{eq:joinH}
		Q'_{1}=Q_{C}[\downarrow\mathit{enum}_{\mathit{Var}'_{\text{thread}}}(1)]
		\dots[\downarrow\mathit{Var}'_{\text{thread}}
		(|\mathit{Var}'_{\text{thread}}|)]
	\end{equation*}
	Note that for $Q'_{1}$ all \emph{\ac{PE}} are \emph{simple} and every not
	yet returned permission caused the $\mathit{BasePerm}$ to be
	\enquote{starred}.
	\begin{equation*}
		Q'_{2} = Q'_{1}[\setminus (E^{Q'_{1}}_{\mathit{RD}}\cup
			E^{Q'_{1}}_{\mathit{RD^{\ast}}})]
	\end{equation*}
	$Q'_{2}$ represents the state where the heap representation of the
	joined process is reduced to the edges of the alterable set.
	Now two transition rules are used to model the semantics of the join
	statement where the first one describes the case with lost read
	permissions and the second one describes the case where all
	read permissions can be completely returned:
	\begin{prooftree}
		\label{prooftree:join}
		\AxiomC{$E^{Q'_{1}}_{\mathit{RD*}}\neq\emptyset$}
		\AxiomC{$H[\downarrow T_{t}]\xRightarrow{N_{T_{t}}\rightarrow Q'_{2}}H'$}
		\BinaryInfC{$(\text{join}(t), H)\rhd(\varepsilon,H')$}
	\end{prooftree}
	Note that $H[\downarrow T_{t}]$ is used which successively drops the process
	identifier in the token $T_{t}$
	\begin{equation*}
		\text{\enquote{$H[\downarrow T_{t}] = H[\downarrow\enum_{T_{t}}(1)]\dots
		[\downarrow\enum_{T_{t}}(|T_{t}|)]$}}
	\end{equation*}
	but preserves $N_{T_{t}}$ in order to replace it by the given production
	rule.
	\begin{prooftree}
		\AxiomC{$E^{H'_{1}}_{\mathit{RD*}}=\emptyset$}
		\AxiomC{$H[\leftarrow T_{t}]\xRightarrow{N_{T_{t}}\rightarrow Q'_{2}}H'$}
		\BinaryInfC{$(\text{join}(t), H)\rhd(\varepsilon,H')$}
	\end{prooftree}
	This concludes the modelling of the semantics of the presented programming
	language and in the following it is dealt with abstraction techniques to
	approach potentially unbounded structures.
