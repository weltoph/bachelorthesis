\section{Permission Model}
	\label{sec:pm}
	As stated above it is locally accounted for which process a selector is
	accessible. The main goal of the analysis is to state that the execution
	of a program is \emph{data race free}. Where data race refers to the
	situation that two processes access a shared value at the same time and at
	least one of the processes alters the value. This implies a dependency of
	both actions with each other. For a proper analysis all interleaving have
	to be examined (see \cite[pp. 77-80]{PrinciplesOfModelChecking} for
	details). \emph{Data race freedom} describes the absence of data races which
	can be guaranteed as long the following two conditions are
	met\footnote{this can be easily derived from the explanation of safe
		concurrency in \cite{SeparationLogic}}:
	\begin{itemize}
		\label{itemize:invariants}
		\item If two or more parallel processes access a value it must not be
			altered.
		\item If one process alters a value it must have at this time exclusive
			access to it.
	\end{itemize}
	The accounting is realised in form of \emph{permissions}. Permissions are
	used to model access restrictions in order to ensure
	\begin{enumerate*}[label=(\arabic*)]
		\item that write access is exclusive and
		\item that if other processes might access the same value at the same
			time both are restricted to read access.
	\end{enumerate*}
	Permissions were already introduced for other modelling techniques for heaps
	like \emph{separation logic} \cite{SeparationLogic}. There different models
	for permissions are examined like \emph{fractional permissions} or
	\emph{counting permissions}. All these models share the property that there
	is one distinguishable permission representing exclusive access on a value.
	From this access ticket for one value multiple other access tickets on this
	value can be derived which then causes all these accesses to become read
	only. Furthermore derived access tickets can be returned until there exists
	only one access tickets which then again grants permission to alter the
	given value.
	For fractional permission it is possible that once derived
	tickets can be split indefinitly on demand to generate as many access
	tickets as needed. On the other hand for counting permissions there is one
	distinguishable permission per value from which further access tickets are
	derived. Every derived access ticket is accounted for and can be returned to
	this distinguishable permission.
	In the following course an approach is chosen which is closer
	connected with \cite{FractionalPermissions} and combines both presented
	concepts of \emph{fractional permissions} and \emph{counting permissions}
	and formalises ideas from \cite{NollJansenDraft}.

	Fundamentally two different \emph{permissions} are distinguished:
	\begin{itemize}
		\item $\mathit{WR}$ - denotes that this value can be altered.
		\item $\mathit{RD}$ - denotes a shared access, which restricts to
			reading only.
	\end{itemize}
	These permissions represent access rights where $\mathit{WR}$ is the
	distinguishable value representing exclusive access. From both permissions
	it is possible to derive further access tickets which are locally
	accounted and deleted if these tickets are returnded. Therefore
	the different forked processes are uniquely identified and this
	identification is used to account derived access tickets. Additionally it is
	possible to transfer $\mathit{WR}$ tickets completely. For the
	identification of forked processes the programming language relies on values
	from $\PI$. As it can be seen from the grammar of the programming language
	it is possible that
	multiple process identifiers from $\PI$ can refer to the same actual process
	as it can be easily seen by the following example:
	\begin{lstlisting}
$t_{1}$ = fork(traverse($\mathit{tmp}$));
$t_{2} = t_{1}$;
join($t_{2}$)
	\end{lstlisting}
	Here $t_{1}$ and $t_{2}$ refer to the same forked process. The join
	statement takes any of these possible process identifier to synchronise with
	the corresponding process. In order to account permissions
	given to processes \emph{tokens} are introduced. Tokens are sets of $\PI$
	which refer to the same process. Thus, for every token $T$ always holds that
	$T\subseteq \PI$ and the set of all possible tokens is simply the powerset
	of the process identifiers ($\mathbb{P}(\PI)$). For every shared value
	the permission that it held initially is kept as well as a set of tokens
	that represent all derived tickets to the process that can access this
	value. Recall that variables and process identifiers are considered process
	exclusive and only heap objects and selectors can be shared. Furthermore the
	presented analysis focuses on structural analysis of the heap and therefore
	only consideres selectors as data of heap objects.
	
	In order to formalises these ideas \emph{\acp{PE}} are introduced in
	the following. These consist of a so called $\mathit{BasePerm}$ which
	represents the \emph{permission} that was originally granted and a set of
	tokens for derived permissions.

	For an example consider Figure \ref{fig:BinTreeProgLangPerm}, which mirrors
	the previous example from Figure \ref{fig:BinTreeProgLang} but added
	permissions to the selectors. Imagine all selectors are initially equipped
	with a $\mathit{WR}$ permission. For the part of the heap that the newly
	forked process may access the token $\{t\}$ is added to the permissions.
	\begin{figure}
	\begin{center}\input{tikz/BinTreeProgLangPerm}\end{center}
		\caption{State of the first iteration of the heap shown in Figure
			\ref{fig:BinTreeProgLang} with additional permissions on every
		selector, empty $\mathit{PermSets}$ are omitted, $\mathit{left}$ and
		$\mathit{right}$ are shortened to $l$ and $r$ respectively. For two
		selectors that are represented by one edge the permission for those is
		written only once.}
		\label{fig:BinTreeProgLangPerm}
	\end{figure}
	As already observed in the second iteration of the example from Figure
	\ref{fig:BinTreeProgLang} the information if a process still operates on a
	certain part of the heap is lost. Especially it is impossible to retrieve
	any information of the status of the once forked process because the
	synchronisation mechanism (join statement) demands an identifier referring
	to the process which no longer exists. In this case the derived access
	tickets are considered permanently lost. In order to model this for
	permissions two additional $\mathit{BasePerm}$ besides
	$\mathit{WR},\mathit{RD}$ are introduced: $\mathit{WR^{\ast}}$ and
	$\mathit{RD^{\ast}}$ which indicate that the initially granted permission
	can never be fully returned since the information about any parallel
	executing processes is lost. The following definition formalises these
	observations:
	\begin{definition}[Permission Expression]
		A \emph{\acl{PE}} is a term of the form
		$$\mathit{BasePerm} - \mathit{PermSet}$$
		where $\mathit{BasePerm}\in
		\{\mathit{WR},\mathit{WR^\ast},\mathit{RD},\mathit{RD^\ast}\}$.
		And $\mathit{PermSet}\subseteq\mathbb{P}(\PI)$.
	\end{definition}
	For convenience an empty $\mathit{PermSet}$ and a $\mathit{PermSet}$
	after $\mathit{RD^\ast}$ or $\mathit{WR^\ast}$ as $\mathit{BasePerm}$ might
	be omitted if it is clear from the context that this information is not
	needed.  Sometimes for a $\mathit{BasePerm}$ $\rho$ and a $\mathit{PermSet}$
	$\Phi$ and $T\in\mathbb{P}(\PI)$
	$\rho - \Phi - T$ is written for $\rho - \Phi\cup\{T\}$.
	In addition a \emph{\ac{PE}} with an empty $\mathit{PermSet}$ is called
	\emph{simple}. The set of all \emph{\acp{PE}} over $\PI$ is denoted by
	$\mathit{PES}_{\PI}$. Note that for a finite set $\PI$ $\mathit{PES}_{\PI}$
	is finite as well.

	For now it was always considered that the process only reads the part of the
	heap it has access to. But as it is introduced later on there also will be a
	way to specify for a process that it might alter certain selectors. In order
	to model that this part of the heap must not be accessed anymore because the
	access ticket is completely submitted to the forked process. This is
	achieved by substituting this part of the heap that might be altered by a
	placeholder.
	This placeholder ``hides'' parts of the heap so it can not be accessed.
	Upon joining this process again the placeholder is additionally used to
	identifiy where the part of the heap to which the access ticket is
	transferred to is returned.
	For the program from Listing \ref{lst:ExampleProgram} consider that the
	forked process might demand write access on the right subtree of its
	parameter. Then the substitution of this subtree by a placeholder can be
	seen in Figure \ref{fig:BinTreeProgLangPermReplacement}. Note that the
	placeholder is connected to the node which can still be accessed from the
	original heap but is also the node that the parameter of $\mathit{traverse}$
	pointed to and to the \textbf{null} node.
	\begin{figure}
	\begin{center}\begin{tikzpicture}[node distance=2 and 2]
	\node[node] (r)  {};
	\node[var] (curr) [above=of r] {$\mathit{curr}$};
	\node[node] (r-l) [below left=of r] {};
	\node[node] (r-r) [below right=of r] {};
	\node[node, node distance=2 and 1.2] (r-l-l) [below left=of r-l] {};
	\node[node, node distance=2 and 1.2] (r-l-r) [below right=of r-l] {};
	\node[node, node distance=2 and 1.2] (r-r-l) [below left=of r-r] {};
	\node[node, node distance=2 and 0.9] (r-l-l-r) [below right=of r-l-l] {};
	\node[he, node distance=2 and 0.9] (Nt) [below right=of r-r] {$N_{\{t\}}$};
	\node[node, below right=2 and 1 of r-l-r] (null) {$v_{\text{null}}$};
	\draw [connector] (curr) to (r);
	\draw [connector] (r-r) to (Nt);
	\draw [connector, bend left=50] (Nt) to (null);
	\draw [sel] (r) to node[left]{$l_{\mathit{WR}}$} (r-l) ;
	\draw [sel] (r) to node[right]{$r_{\mathit{WR}}$} (r-r);
	\draw [sel] (r-l) to node[left]{$l_{\mathit{WR}}$} (r-l-l);
	\draw [sel] (r-l) to node[right]{$r_{\mathit{WR}}$} (r-l-r);
	\draw [sel] (r-r) to node[right] {$l_{\mathit{WR} - \{\{t\}\}}$} (r-r-l);
	\draw [sel] (r-l-l) to node[right]{$r_{\mathit{WR}}$} (r-l-l-r);
	\draw [sel] (r-l-l) to node[above]{$l_{\mathit{WR}}$} (null);
	\draw [sel] (r-l-r) to node{$(l,r)_{\mathit{WR}}$} (null);
	\draw [sel] (r-r-l) to node[above right]{$(l,r)_{\mathit{WR} - \{\{t\}\}}$}
		(null);
	\draw [sel, bend right] (r-l-l-r) to node[below]{$(l,r)_{\mathit{WR}}$}
		(null);

	\begin{scope}[on background layer]
		\node[fill=green!30,fit=(r-r) (r-r-r) (r-r-l) (r-r-r-r),label=125:$t$]{};
	\end{scope}
\end{tikzpicture}

\end{center}
		\caption{State of the first iteration of the heap shown in figure
			\ref{fig:BinTreeProgLang} with permissions as well as partially
		substitution with a placeholder.}
		\label{fig:BinTreeProgLangPermReplacement}
	\end{figure}
