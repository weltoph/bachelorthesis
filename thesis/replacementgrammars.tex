\section{Concretisation \& Abstraction}
	Making use of dynamic allocation and deallocation can lead to heap
	structures of unbounded size. In order to present these structures in a
	finite manner \emph{\ac{HR}} can be used in order to concretise and abstract
	heap states by applying production rules for concretisation and by applying
	production rules backwards for abstraction. Therefore introduce a set of
	production rules in order to abstract parts of the heap representation.
	In order to illustrate the
	following steps the data structure of binary trees is examined along the
	definitions\footnote{this example reassembles the \emph{\acl*{HRG}} from
	\cite[p. 21]{fmsd}}. Reconsider the structure in the example from Figure
	\ref{fig:BinTreeProgLang} and furthermore the \emph{\ac{HR}} from Figure
	\ref{fig:replacement}. When the node marked by $n$ is interpreted as
	$v_{\text{null}}$ and the production rule from Figure
	\ref{fig:productionrule} is applied repetitively arbitrary large binary
	trees can be constructed. But it is already noteable that repetitive
	application of this production rule always leads to further $B$-labeled
	edges. The following formal introduction of the concept of abstraction
	follows mostly \cites{fmsd}{InductivePredicates}{LocalGreibachNormalForm}.

	In order to distinguish the hyperedges that are used for the abstract
	representation of a subgraph from the hyperedges that represent
	placeholders, selectors or variables a new set of so called
	\emph{nonterminals} $N$ is introduced. For the above considered example of
	binary trees this set would be $N=\{B\}$. In the following
	$\Sigma_{N} = \Sigma\biguplus N$ denotes the set of labels for
	\emph{\acp{HG}} that are partially concrete and partially abstract.
	Furthermore these \emph{nonterminals} are expected to connect always the
	same amount of nodes. This amount is called the rank of the nonterminal.
	Therefore a function $\rk\colon N\rightarrow \mathbb{N}$ is given and for
	every \emph{\ac{HG}} $H$ it holds that if $\lab_{H}(e)\in N$ then
	$|\con_{H}(e)|=\rk(\lab_{H}(e))$, the rank of $B$ for example is $2$.
	The set of all these partially abstract \emph{\acp{HG}} is denoted by $\HGN$
	and the set of \emph{\acp{HG}} that furthermore satisfy the restrictions of
	\emph{\acp{HC}} is denoted by $\HCN$. Note that for now the permissions were
	neglected but are actually incorporated as follows: a production rule
	$p\colon X\rightarrow H$ with $X\in N$ is annotated by
	a permission $\rho$ ($p_{\rho}$) such that for
	$p_{\rho}\colon X\rightarrow H$
	holds that $\img(\perm_{H}) = \{\rho\}$ (thus all edges in $H$ hold
	permission $\rho$). Furthermore this production rule must only be applied to
	an $X$-labeled edge if this edge holds the permission $\rho$.

	\subsection{Hyperedge Replacement Grammar}
	To apply abstraction and concretisation a set of production rules is used.
	Such a set of production rules is called a \emph{\ac{HRG}}. Formally defined
	as follows:
	\begin{definition}[Hyperedge Replacement Grammar]
		A \emph{\acl{HRG}} over an alphabet $\Sigma_N$ and
		$\PI$ is a finite set of
		production rules of the form $p_{\rho}\colon X\rightarrow G$ where
		$X\in N$, $\rho\in\mathit{PES}_{\PI}$,
		$G\in\mathit{HG}_{\Sigma_{N}}$ and $\rk(X) = |\ext_{G}|$.
	\end{definition}
	$\HRG$ denotes the set of all \emph{\acp{HRG}} over $\Sigma_{N}$ and
	$\PI$. Especially, in this paper the concepts of structural abstraction and
	permission accounting are independently examined, this means in order to
	separate those concepts the following definition is used which states that
	every production rule is available for every permission:
	\begin{definition}[Fully Permissive Grammar]
		\label{def:fpg}
		$G\in\HRG$ is called \emph{fully permissive} if the following holds:
		Every production rule $p\colon X\rightarrow H$ in $G$ exists for every
		$\rho\in\PES$.
	\end{definition}
	The set of all \emph{\acp{fpHRG}} is denoted by $\fpHRG$.

	In the following some definitions regarding \emph{\acp{fpHRG}} are
	introduced. These intuitions and their respective definitions are examined
	in their structural concepts only since the incorporation of permissions is
	straightforward:
	\begin{itemize}
		\item For $G\in\fpHRG$ $G^{X} = \{ X'\rightarrow H\in G \mid X'=X\}$
			denotes the set of all production rules for a nonterminal $X$.
		\item For a production rule $p\colon X\rightarrow H$ $\lhs(p)$ is used
			for $X$ and respectively $\rhs(p)$ for the right hand side $H$. $\rhs$
			and $\lhs$ are lifted to sets of \emph{production rules} by
			elementwise application.
		\item The \emph{handle} of a nonterminal is introduced which is
			the \emph{\ac{HG}} that contains one edge labeled by the nonterminal
			and nodes connected to this nonterminal. Formally:
				\begin{definition}[Handle]
					For a \emph{\ac{fpHRG}} $G\in\fpHRG$ and a nonterminal $X\in N$
					the \emph{handle} of $X$ is the \emph{\ac{HG}} $X^{\bullet}
					\in\mathit{HG}_{\Sigma_{N}}$
					with:
					\begin{itemize}
						\item $V_{X^{\bullet}} = \{v_1,\dots,v_{\mathit{rk}(X)}\}$
						\item $E_{X^{\bullet}} = \{e\}$
						\item $\mathit{con}_{X^{\bullet}} =
							\{e\mapsto v_1\cdots v_{\mathit{rk}(X)}\}$
						\item $\mathit{lab}_{X^{\bullet}} =
							\{e\mapsto X\}$
						\item $\mathit{ext}_{X^{\bullet}} = \epsilon$
						\item $\mathit{perm}_{X^{\bullet}} = \{e\mapsto\rho\}$
					\end{itemize}
				\end{definition}
		\item For hyperedges labeled with nonterminals the position on
			which a node is attached on is called a \emph{tentacle}. A
			tentacle is just a label and the position a node is attached to an
			edge of this label.
				\begin{definition}[Tentacle]
					A \emph{tentacle} is a pair $(X,i)$ where $X\in N$ and
					$i\in\{1,\dots \mathit{rk}(X)\}$.
				\end{definition}
		\item For hyperedges labeled with nonterminals the notion of
			tentacles can be extended in order to describe how two nodes are
			connected by this hyperedge. Such a connection is called a
			\emph{bridge} and reassembles two tentacles of the same nonterminal.
				\begin{definition}[Bridge]
					A \emph{bridge} is a triple $(i, X, j)$ where $X\in N$ and
					$1\leq i,j\leq \rk(X)$.
				\end{definition}
			Let $\br(S)$ denote the set of all bridges for the set of ranked
			labels $S$.
	\end{itemize}
	Consider the following possible \emph{\ac{fpHRG}} $G$ which describes the
	abstraction of binary trees:
	\begin{equation*}
		\label{eq:G}
		G = \left\{\begin{aligned}
			 & \parbox{5cm}{\resizebox{5cm}{!}{
					\begin{tikzpicture}
	\node[he] (lhs) {$B$};
	\node[node, right=1cm and 2cm of lhs] (v2) {};
	\node[node, ext, above right=of v2] (v1) {$1$};
	\node[node, below right=of v1] (v3) {};
	\node[node, below right=of v2] (null) {$v_{\text{null}}$};

	\draw[very thick, ->, shorten >=0.5cm, shorten <=0.5cm] (lhs) to (v2);
	\draw[sel] (v1) to node[left] {$l$} (v2);
	\draw[sel] (v1) to node[right]{$r$} (v3);
	\draw[sel, bend left] (v2) to node[above]{$l$} (null);
	\draw[sel, bend right] (v2) to node[below]{$r$} (null);
	\draw[sel, bend left] (v3) to node[below]{$l$} (null);
	\draw[sel, bend right] (v3) to node[above]{$r$} (null);
\end{tikzpicture}

				}}, \parbox{5cm}{\resizebox{5cm}{!}{
					\begin{tikzpicture}
	\node[he] (lhs) {$B$};
	\node[node, right=1cm and 2cm of lhs] (v2) {};
	\node[node, ext, above right=of v2] (v1) {$1$};
	\node[node, below right=of v1] (v3) {};
	\node[node, below right=of v2] (null) {$v_{\text{null}}$};
	\node[he] (e1) at (null-|v2) {$B$};

	\draw[very thick, ->, shorten >=0.3cm, shorten <=0.5cm] (lhs) to (v2);
	\draw[sel] (v1) to node[left] {$l$} (v2);
	\draw[sel] (v1) to node[right]{$r$} (v3);
	\draw[connector] (v2) to node[left]{$1$} (e1);
	\draw[sel, bend left] (v3) to node[below]{$l$} (null);
	\draw[sel, bend right] (v3) to node[above]{$r$} (null);
	\draw[connector] (e1) to (null);
\end{tikzpicture}

				}},\\& \parbox{5cm}{\resizebox{5cm}{!}{
					\begin{tikzpicture}
	\node[he] (lhs) {$B$};
	\node[node, right=1cm and 2cm of lhs] (v2) {};
	\node[node, ext, above right=of v2] (v1) {$1$};
	\node[node, below right=of v1] (v3) {};
	\node[node, below right=of v2] (null) {$v_{\text{null}}$};
	\node[he] (e1) at (null-|v3) {$B$};

	\draw[very thick, ->, shorten >=0.3cm, shorten <=0.5cm] (lhs) to (v2);
	\draw[sel] (v1) to node[left] {$l$} (v2);
	\draw[sel] (v1) to node[right]{$r$} (v3);
	\draw[connector] (v3) to node[left]{$1$} (e1);
	\draw[sel, bend left] (v2) to node[above]{$l$} (null);
	\draw[sel, bend right] (v2) to node[below]{$r$} (null);
	\draw[connector] (e1) to (null);
\end{tikzpicture}

				}}, \parbox{5cm}{\resizebox{5cm}{!}{
					\begin{tikzpicture}
	\node[he] (lhs) {$B$};
	\node[node, right=1cm and 2cm of lhs] (v2) {};
	\node[node, ext, above right=of v2] (v1) {$1$};
	\node[node, below right=of v1] (v3) {};
	\node[he, below=of v3] (e2) {$B$};
	\node[he, below=of v2] (e1) {$B$};
	\node[node] (null) at (e1-|v1) {$v_{\text{null}}$};

	\draw[very thick, ->, shorten >=0.3cm, shorten <=0.5cm] (lhs) to (v2);
	\draw[sel] (v1) to node[left] {$l$} (v2);
	\draw[sel] (v1) to node[right]{$r$} (v3);
	\draw[connector] (v3) to node[left]{$1$} (e2);
	\draw[connector] (v2) to node[right]{$1$} (e1);
	\draw[connector] (e1) to (null);
	\draw[connector] (e2) to (null);
\end{tikzpicture}

		}}
		\end{aligned} \right\}
	\end{equation*}
	Note that these production rules are present for every permission, since it
	is a fully permissive \emph{\ac{HRG}} and that from $H\in\HGN$ it can be
	possible to derive multiple \emph{\acp{HG}} since there might be
	more than one production rule for any nonterminal. Consider
	therefore the \emph{\ac{HG}} of Figure \ref{fig:replacement} and the
	\emph{\ac{fpHRG}} $G$ from above. It is easily to see that multiple
	different \emph{\acp{HG}} can be derived. The set of all these deriveable
	\emph{\acp{HG}} which no longer contain a nonterminal (like the terminal
	words of a string grammar) is called the language of $H$ and defined as
	follows:
	\begin{definition}[Language of an \ac{HG}]
		For $G\in\fpHRG$ and $H\in\mathit{HG}_{\Sigma_{N}}$
		\begin{equation*}
			L_{G}(H) = \{K\in\mathit{HG}_{\Sigma} \mid H \Rightarrow^{*} K\}
		\end{equation*}
		is the \emph{language} from the \emph{\ac{HG}} $H$
		(under the \emph{\ac{fpHRG}} $G$).
	\end{definition}

	\subsection{Properties}
	In the following some properties of \emph{\acp{HRG}} are discussed. The
	presented properties as well as the presented results for \emph{\acp{fpHRG}}
	follow from \cite{LocalGreibachNormalForm}. Actually the design of the
	definitions are built around those in \cite{LocalGreibachNormalForm} to
	focus on the compatibility of adding permissions to the already established
	results. At first it is discussed which selectors a nonterminal actually
	abstracts. By considering the example in Figure \ref{fig:replacement} it can
	easily be seen, that at the different connected nodes of the $B$-labeled
	edge different selectors are abstracted. At the node on the tentacle $(B,1)$
	both selectors $l$ and $r$ are \enquote{hidden} in the abstraction of the
	hyperedge whereas the node on tentacle $(B,2)$ identifies $v_{\text{null}}$
	which must not have any outgoing selectors. This illustrates that different
	selectors can be abstracted at different tentacles. \emph{Typedness} is
	introduced to ensure that application production rules reveals the same
	selectors for every tentacle. The function $\type$ maps from tentacles to
	the sets of the selectors this tentacle abstracts. In the \emph{\ac{fpHRG}}
	from page \pageref{eq:G} it holds that $\type(B,1) = \{l,r\}$. The formal
	definition is as follows:
	\begin{definition}[Typedness]
		For $G\in\fpHRG$ a nonterminal $X\in N$ is called typed if for all
		tentacles $(X,i)$ there is a set $\type(X,i)\subseteq\Sel$ such that for
		all $H\in L_{G}(X^{\bullet})$ where $v^{\bullet}_{i}$ is the node of the
		tentacle $(X,i)$ in $X^{\bullet}$ it holds:
		\begin{equation*}
			\type(X,i) = \lab_H(\out_{H}(v^{\bullet}_{i}))
		\end{equation*}
	\end{definition}
	Note that $\out_{H}(v) = \{e \in E_{H}\mid \con_{H}(1) = v
	\land \lab_{H}(e)\in\Sel\}$ is the set of all outgoing
	selectors from the object represented by $v$. Since different tentacles
	might have different types but the definition
	of \emph{\ac{HC}} enforces that every node is universally typed it is
	possible that the nodes of a handle violate this typing. Therefore the
	notion of handles is expanded to ensure that at least the initial nodes of
	the handle are universally typed:
	\begin{definition}[Typed Handle]
		For a typed nonterminal $X\in N$ let $\mathit{Type} =
		\bigcup_{1\leq i\leq\rk(X)}\type(X,i)$ denote the set of all selectors of
		the nonterminal. The typed handle of $X$ (denoted as $X^{\circ}$) is
		defined as follows:
		\begin{itemize}
			\item $V_{X^{\circ}} = \{v_{1},\dots,v_{\rk(X)},v_{\text{null}}
				\}$
			\item $E_{X^{\circ}} = \{e\} \uplus \{e_{s,i}\mid 1\leq i\leq
				\rk(X), s\in\mathit{Type}\setminus\type(X,i)\}$
			\item $\con_{X^{\circ}} = \{e\mapsto v_{1}\cdots v_{\rk(X)}\}
				\uplus\{e_{s,i}\mapsto v_{i}v_{\text{null}}\mid e_{s,i}\in
				E_{X^{\circ}}\}$
			\item $\lab_{X^{\circ}} = \{e\mapsto X\}\uplus\{e_{s,i}\mapsto s
				\mid e_{s,i}\in E_{X^{\circ}}\}$
			\item $\ext_{X^{\circ}} = \varepsilon$
			\item $\perm_{X^{\circ}} = \{e\mapsto\mathit{WR}\mid e\in
				E_{X^{\circ}}\}$
		\end{itemize}
	\end{definition}

	Secondly, \emph{productivity} is examined which simply states that at
	least one terminal \emph{\ac{HG}} can be derived from this nonterminal.
	Formally stated as:
	\begin{definition}[Productivity]
		For $G\in\HRG$ a nonterminal $X\in N$ is called \emph{productive} if
		\begin{equation*}
			L_{G}(X^{\bullet})\neq\emptyset
		\end{equation*}
		$G$ is called \emph{productive} if all nonterminals in $G$ are
		\emph{productive}.
	\end{definition}

	Following \emph{increasingess} is introduced. The idea is that every right
	hand side of a production rule is strictly \enquote{bigger} than the left
	hand side. Where \enquote{bigger} refers to the number of edges, but also
	terminal \emph{\acp{HG}} are considered \enquote{bigger}. This ensures that
	applying succesively production rules in a backward fashion terminates,
	since every applied production rules reduces the size of the \emph{\ac{HG}}.
	\begin{definition}[Increasingess]
		For $G\in\fpHRG$ a production rule $p\colon X\rightarrow H\in G$ is
		called \emph{increasing} if $H\in\HG$ or $H\in\aHG \land |E_{H}| > 1$.
		$G$ is called \emph{increasing} if all $p\in G$ are increasing.
	\end{definition}

	Fourthly, \emph{local concretisability} means that for every tentacle there
	are production rules that reveal abstracted selectors but preserve the
	language of the graph. In order to illustrate the problem another
	\emph{\ac{fpHRG}} is introduced\footnote{this example is featured in
		various contributions to this approach \cites{InformalGraphGrammars}
		{InductivePredicates} because of its illustrating quality}:
	\begin{equation*}
		\label{eq:G'}
		G' = \left\{
			\begin{aligned}
				&\begin{tikzpicture}
	\node[he] (lhs) {$L$};
	\node[node, ext, right=1cm and 2cm of lhs] (v1) {$1$};
	\node[node, ext, right=of v1] (v2) {$2$};
	
	\draw[->, very thick, shorten >=0.2cm, shorten <=0.2cm] (lhs) to (v1);
	\draw[sel, bend left] (v1) to node[above]{$\mathit{next}$} (v2);
	\draw[sel, bend left] (v2) to node[below]{$\mathit{prev}$} (v1);
\end{tikzpicture}
,\\
				&\begin{tikzpicture}
	\node[he] (lhs) {$L$};
	\node[node, ext, right=1cm and 2cm of lhs] (v1) {$1$};
	\node[node, right=of v1] (v2) {};
	\node[he, right=of v2] (e) {$L$};
	\node[node, ext, right=of e] (v3) {$2$};
	
	\draw[->,very thick, shorten >=0.2cm, shorten <=0.2cm] (lhs) to (v1);
	\draw[sel, bend left] (v1) to node[above]{$\mathit{next}$} (v2);
	\draw[sel, bend left] (v2) to node[below]{$\mathit{prev}$} (v1);
	\draw[connector] (e) to node[above]{$1$} (v2);
	\draw[connector] (e) to node[above]{$2$} (v3);
\end{tikzpicture}

			\end{aligned}
		\right\}
	\end{equation*}
	This \emph{\ac{fpHRG}} describes doublely linked list of arbitrary length.
	Firstly, it holds that
	$\type(L,1) = \{\mathit{next}\}$ and $\type(L,2) = \{\mathit{prev}\}$. But,
	consider the typed handle of $L$, where $n$, $p$ abbreviate $\mathit{next}$,
	$\mathit{prev}$ respectively:
	\begin{center}
		\begin{tikzpicture}
	\node[node] (v1) {$v_{1}$};
	\node[he, right=of v1] (L) {$L$};
	\node[node, right=of L] (v2) {$v_{2}$};
	\node[node, below=of L] (null) {$v_{\text{null}}$};

	\draw[connector] (L)  to node[above]{$1$} (v1)  ;
	\draw[connector] (L)  to node[above]{$2$} (v2)  ;
	\draw[sel]       (v1) to node[left] {$p$} (null);
	\draw[sel]       (v2) to node[right]{$n$} (null);

\end{tikzpicture}

	\end{center}
	In order to reveal the abstracted selectors at the tentacle $(L,2)$ only
	the first production rule can be applied. But this reduces the language to
	the singleton set of the right hand side of the first production rule.
	Repeatedly applying the second production rule does not reveal the
	abstracted selectors. Thus this grammar is not locally concretisable.
	Adding the production rule
	\begin{equation*}
		\label{eq:listrule3}
		\begin{tikzpicture}
	\node[he] (lhs) {$L$};
	\node[node, ext, right=1cm and 2cm of lhs] (v1) {$1$};
	\node[he, right=of v1] (e) {$L$};
	\node[node, right=of e] (v2) {};
	\node[node, ext, right=of v2] (v3) {$2$};
	
	\draw[->,very thick, shorten >=0.2cm, shorten <=0.2cm] (lhs) to (v1);
	\draw[sel, bend left] (v2) to node[above]{$\mathit{next}$} (v3);
	\draw[sel, bend left] (v3) to node[below]{$\mathit{prev}$} (v2);
	\draw[connector] (e) to node[above]{$1$} (v1);
	\draw[connector] (e) to node[above]{$2$} (v2);
\end{tikzpicture}

	\end{equation*}
	establishes local concretisability because it is now possible to reveal
	abstracted selectors at tentacle $(L,2)$ but the generated language (doubly
	linked lists of arbitrary length) is preserved.
	
	The formal definition of local concretisability requires some additional
	defintions, such as: $(X, i)\rightarrow_{p}(Y,j)$ denotes that for a
	production rule $p\colon X\rightarrow H$ there is an edge $e\in E_{H}$ such that
	$\lab_{H}(e) = Y$ and $\ext_{H}(i) = \con_{H}(e)(j)$. This means that after
	applying the production rule $p$ the tentacle $(X, i)$ morphs (for the
	connected node) to a tentacle $(Y, j)$. Furthermore let $\overline{G^{X}}$
	denote the set of production rules in a \emph{\ac{fpHRG}} $G$ for all
	nonterminals except of $X$, i.e. $\overline{G^{X}} = G\setminus G^{X}$. Then
	local concretisability can be defined as follows:
	\begin{definition}[Local Concretisability]
		$G\in\fpHRG$ is called \emph{local concretisable} if for all nonterminals
		$X$ there are sub-grammars $G_{1},\dots,G_{\rk(X)}\subseteq G$ such
		that $\forall i\in\{1,\dots,\rk(X)\}.L_{G_{i}^{X}\cup\overline{G^{X}}}
		(X^{\bullet})=L_{G}(X^{\bullet})$ and $\forall i\in\{1,\dots,\rk(X)\}.
		\forall p\in G_{i}^{X}.(X,i)\rightarrow_{p}(a,1)$ for all selectors $a$
		in $\type(X,i)$.
	\end{definition}

	The following definition gathers now all \emph{\ac{fpHRG}} for which the
	language of the typed handle contains only valid \emph{\ac{HC}}. This
	ensures that not external nodes of the right hand sides are properly typed
	and follow the properties of \emph{\acp{HC}}.
	\begin{definition}[Data Structure Grammar]
		A \emph{\ac{HRG}} $G\in\fpHRG$ is called a \emph{\ac{DSG}}
		if it is typed and for all \emph{nonterminals} $X\in N$ the language of
		its typed handle $X^{\circ}$ only contains valid \emph{\aclp{HC}}
		($L_{G}(X^{\circ})\subseteq\HC$) and for all inner nodes of right hand
		sides in $G^{X}$ it holds that there is one outgoing edge for every
		selector in $\bigcup_{1\leq i\leq\rk(X)}\type(X,i)$.
	\end{definition}
	Note that it is assumed that only the selectors exist which are gathered
	in the different types of the tentacles of the nonterminal. If more
	selectors exist they can be simply attached to every node and point to
	$v_{\text{null}}$ to ensure that the \emph{\acp{HG}} of the language are
	still valid \emph{\acp{HC}}. Again let $\DSG$ denote the set of all
	\emph{\acp{DSG}} over $\Sigma_{N}$ and $\PI$. Additionally, it is generally
	assumend that right hand sides of production rules do not contain any edges
	labeled with variables or placeholders (thus, $\rhs(G)\subseteq
	\mathit{HG}^{\PI}_{\Sigma_{N}\setminus (\Var\cup\mathbb{T})}$). And
	therefore all inner nodes are properly typed which means that they have
	edges for all selectors that exist within the context of the nonterminal
	(i.e. the union of the types of all tentacles).
	In the following \emph{\aclp*{HAG}} are defined which are \emph{\acp{DSG}}
	that additionally provide the properties introduced above:
	\begin{definition}[Heap Abstraction Grammar]
		$G\in\DSG$ is called a \emph{\ac{HAG}} over $\Sigma_{N}$ if $G$ provides
		the following properties:
		\begin{enumerate}
			\item \emph{Productivity}
			\item \emph{Increasingness}
			\item \emph{Local Concretisability}
		\end{enumerate}
	\end{definition}
	This is no real restriction since the following theorem can be derived from
	the results of \cite{LocalGreibachNormalForm}:
	\begin{theorem}
		For every \emph{\ac{DSG}} a \emph{\ac{HAG}} can be constructed which is
		equivalent regarding the language of \emph{\acp{HG}}.
	\end{theorem}

