\section{Hypergraphs as Heap Representation}
	In order to formalise the intuitive graph representation from the examples
	above \emph{\acp{HG}} are introduced in the following. This follows the
	general approach presented in \cites{LocalGreibachNormalForm}
	{InductivePredicates}{InformalGraphGrammars}{fmsd}. Intuitively, an
	\emph{\ac{HG}} is the same as a normal graph, it consists of a set of nodes
	and edges, but an edge in an \emph{\ac{HG}} is called hyperedge and can
	connect arbitrarily many nodes instead of only two. Furthermore the used
	\emph{\acp{HG}} will be labeled, thus there is a function $\lab$ which
	assigns a label to every hyperedge. The possible labels come from a set
	$\Sigma$.  Then an \emph{\ac{HG}} can be formally defined as follows:
	\begin{definition}[Hypergraph]
		Let $\Sigma$ be a set of labels and let $V$ be a finite
		set of nodes, $E$ a finite set of edges.
		$\con: E\rightarrow V^{*}$ maps every edge to the sequence of
		connected nodes, $\lab: E\rightarrow\Sigma$ labels every edge.
		Further let $ext\in V^{*}$ denote a possibly empty
		sequence ($\varepsilon$) of nodes and $\perm: E\rightarrow \PES$. An
		\emph{\ac{HG}} $H$ is a tupel
		\begin{equation*}
			H=(V, E, \con, \lab, \ext, \perm)
		\end{equation*}
	\end{definition}
	The set of all hyperedges with the labels $\Sigma$ and process identifier 
	$\PI$ is denoted by $\HG$. Two \emph{\acp{HG}} are called \emph{isomorphic}
	if they are equivalent modulo renaming of edges and nodes. Isomorphic
	\emph{\acp{HG}} are not distinguished in the following.

	In order to represent states of the heap the objects in the heap will be
	represented as nodes. Variables are represented by hyperedges that are
	only connected to the node that represents the value of the variable. Such
	a hyperedge that is connected to only one node is called \emph{unary}.
	Selectors are modelled as binary hyperedges between nodes, which are
	interpreted as directed from the first node of the connection sequence to
	the second one. Furthermore parts that are given away to other processes
	are substituted by hyperedges. For the hyperedges that represent parts of
	the heap for which access tickets are transferred a set of labels is
	introduced as
	\begin{equation*}
		\mathbb{T} = \{N_{T} \mid T\in\mathbb{P}(\PI)\}
	\end{equation*}
	States of the heap can now be represented as $H\in\HG$ where
	$\Sigma = \Var\uplus\Sel\uplus\mathbb{T}$, but of course not all those
	\emph{\ac{HG}} $H$ represent valid heaps because it is e.g. possible that
	every object might have multiple outgoing edges labeled with the same
	selector. Thus some restrictions are enforced in order to represent valid
	states of a heap which are called \emph{\acp{HC}}, namely:
	\begin{itemize}
		\item There is exactly one distinguishable node $v_{\text{null}}$
			representing the \textbf{null} value. $v_{\text{null}}$ has no
			outgoing selectors.
		\item There is exactly one unary hyperedge for every variable.
		\item Edges labeled by a selector are binary hyperedges.
		\item For every node there is at most one outgoing edge per
			selector.
	\end{itemize}
	Note that every object is universally typed. Thus, all objects have
	initially an outgoing edge for every selector. But of course some of these
	edges
	can be \enquote{hidden} within a hyperedge representing parts of the heap
	for which $\mathit{WR}$ permissions are transferred to another process. As
	it is possible that the reference to the process that operates exclusively
	on the selectors is lost, these selectors can permanently be removed from
	the heap state.
	This leads to the following formal definition for \emph{\acp{HC}}:
	\begin{definition}[Heap Configuration]
		A \emph{\acl{HG}}\\
		$H = (V, E, \mathit{con}, \mathit{lab}, \mathit{ext}, \mathit{perm}) \in
		\HG$ is called a \emph{\acl{HC}} if
		\begin{itemize}
			\item $\Sigma = \mathit{Var}\biguplus\mathit{Sel}\biguplus \mathbb{T}$
			\item $v_{\text{null}}\in V$ and
				$\nexists e\in E.\lab_{H}(e)\in\Sel\land
				\con_{H}(e)(1)=v_{\text{null}}$
			\item $|\con_{H}(e)| = 1$ for all $e\in E$ with $\lab_{H}(e)\in
				\mathit{Var}$
			\item $|\con_{H}(e)| = 2$ for all $e\in E$ with $\lab_{H}(e)\in
				\mathit{Sel}$
			\item $\forall v\in\mathit{Var}.\exists e\in E. \lab_{H}(e) = v$
			\item $\forall v\in\mathit{Var}, e,e'\in E.
				v = \lab_{H}(e) = \lab_{H}(e') \rightarrow e = e'$
			\item $\forall s\in\mathit{Sel}, e,e'\in E.
				\con_{H}(e)(1) = \con_{H}(e')(1) \land
				s = \lab_{H}(e) = \lab_{H}(e') \rightarrow e = e'$
		\end{itemize}
	\end{definition}
	The set of all \emph{\ac{HC}} over $\Sigma$ and
	$\PI$ is denoted by $\HC$.

	Furthermore the following notion is defined for $\rho\in\PES$ which
	collects all edges of the \ac{HG} with permission $\rho$:
	\begin{equation*}
		E^{H}_{\rho} = \{e\in E_{H}\mid\perm_{H}(e) = \rho\}
	\end{equation*}
