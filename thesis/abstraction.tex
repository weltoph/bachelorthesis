\subsection{Concretisation}
	The basic idea of abstraction is to represent multiple heaps in one
	\emph{\ac{HG}}. Therefore \emph{\acp{HAG}} are used. Recall the
	initial programming example in Listing \ref{lst:ExampleProgram} where one
	possible execution is represented in Figure \ref{fig:BinTreeProgLang}.
	Analysing executions on binary trees of arbitrary size is realised by
	representing these as one single $B$-labeled hyperedge as follows:
	\begin{center}
		\begin{tikzpicture}
	\node[var]                 (curr) {$\mathit{curr}$};
	\node[node, below=of curr] (v)    {};
	\node[node, below left=of v] (null) {$v_{\text{null}}$};
	\node[he, label={0:{$\mathit{WR}$}}] (e) at (null-|v) {$B$};

	\draw[connector] (curr) to                  (v);
	\draw[connector] (e)    to node[right]{$1$} (v);
	\draw[connector] (e)    to                  (null);
\end{tikzpicture}

		\label{tikz:abstractedbintree}
	\end{center}
	where the \emph{\ac{fpHRG}} $G$ from page \pageref{eq:G} (which is actually
	a \emph{\ac{HAG}}) is used to derive the concrete binary trees which are
	represented by the nonterminal $B$. By applying the different production
	rules of $G$ all possible binary trees can be derived. Especially since all
	possible binary trees need to be analysed it is necessary that \emph{every}
	production rule is used to concretise the nonterminal. This leads to four
	possible representations, namely the four right hand sides of the production
	rules in $G$ (as it can be seen in Figure \ref{fig:bintreeconc}).
	\begin{figure}
		\begin{center}
			\begin{tikzpicture}
				\node (1)             {\begin{tikzpicture}
	\node[node] (v1) {$1$};
	\node[node, below left= of v1] (v2) {};
	\node[node, below right=of v1] (v3) {};
	\node[var, above=of v1] (curr) {$\mathit{curr}$};

	\draw[connector] (curr) to (v1);
	\draw[sel] (v1) to node[left] {$l_{\mathit{WR}}$} (v2);
	\draw[sel] (v1) to node[right]{$r_{\mathit{WR}}$} (v3);
\end{tikzpicture}
};
				\node[right=of 1] (d1) {};
				\node[right=of d1] (2) {\begin{tikzpicture}
	\node[var] (curr) {$\mathit{curr}$};
	\node[node, below=of curr] (v1) {$1$};
	\node[node, below left= of v1] (v2) {};
	\node[node, below right=of v1] (v3) {};
	\node[he, below=of v2, label=180:{$\mathit{WR}$}] (e1) {$B$};
	\node[node, below=2.43 and 2 of v1] (null) {$v_{\text{null}}$};

	\draw[sel] (v1) to node[left] {$l_{\mathit{WR}}$} (v2);
	\draw[sel] (v1) to node[right]{$r_{\mathit{WR}}$} (v3);
	\draw[connector] (v2) to node[left]{$1$} (e1);
	\draw[connector] (e1) to (null);
	\draw[connector] (curr) to (v1);
	\draw[sel, bend right] (v3) to node[left]{$l_{\mathit{WR}}$} (null);
	\draw[sel, bend left] (v3) to node[right]{$r_{\mathit{WR}}$} (null);
\end{tikzpicture}
};
				\node[below=of 1] (d3) {};
				\node[below=of d3] (3) {\begin{tikzpicture}
	\node[var] (curr) {$\mathit{curr}$};
	\node[node, below=of curr] (v1) {$1$};
	\node[node, below left=of v1] (v2) {};
	\node[node, below right=of v1] (v3) {};
	\node[he, below=of v3, label=0:{$\mathit{WR}$}] (e1) {$B$};
	\node[node, below=2.44 and 2 of v1] (null) {$v_{\text{null}}$};

	\draw[sel] (v1) to node[left] {$l_{\mathit{WR}}$} (v2);
	\draw[sel] (v1) to node[right]{$r_{\mathit{WR}}$} (v3);
	\draw[connector] (v3) to node[right]{$1$} (e1);
	\draw[connector] (e1) to (null);
	\draw[connector] (curr) to (v1);
	\draw[sel, bend left] (v2) to node[right]{$l_{\mathit{WR}}$} (null);
	\draw[sel, bend right] (v2) to node[left]{$r_{\mathit{WR}}$} (null);
\end{tikzpicture}
};
				\node (d2) at (d3-|2) {};
				\node (4) at (3-|2)   {\begin{tikzpicture}
	\node[var] (curr) {$\mathit{curr}$};
	\node[node, below=of curr] (v1) {$1$};
	\node[node, below left=of v1] (v2) {};
	\node[node, below right=of v1] (v3) {};
	\node[he, below=of v3, label=0:{$\mathit{WR}$}] (e2) {$B$};
	\node[he, below=of v2, label=180:{$\mathit{WR}$}] (e1) {$B$};
	\node[node] (null) at (e2-|curr) {$v_{null}$};

	\draw[sel] (v1) to node[left] {$l_{\mathit{WR}}$} (v2);
	\draw[sel] (v1) to node[right]{$r_{\mathit{WR}}$} (v3);
	\draw[connector] (v3) to node[right]{$1$} (e2);
	\draw[connector] (v2) to node[left] {$1$} (e1);
	\draw[connector] (e1) to (null);
	\draw[connector] (e2) to (null);
	\draw[connector] (curr) to (v1);
\end{tikzpicture}
};
				\node (d4) at (4-|d1) {};

				\draw (d1) to (d4);
				\draw (d2) to (d3);
			\end{tikzpicture}
		\end{center}
		\caption{First step of concretisation for abstraction of binary trees,
			note that the permission of the nonterminal propagates to the inserted
		right hand side}
		\label{fig:bintreeconc}
	\end{figure}
	For the three \emph{\acp{HC}} that still contain nonterminals
	further concretisation steps have to be applied. But since the given
	programming language (see section \ref{sec:syntax}) allows only
	dereferences of depth one (like $x.s$) it is possible to execute all
	statements (besides fork and join) on these partially abstract
	\emph{\acp{HC}} just like they are executed on fully concrete
	\emph{\acp{HC}}. These states of partially abstract \emph{\acp{HC}} are
	called \emph{admissible} and ensure that all actually from statements
	referenceable selectors are present in the \emph{\ac{HC}}. For the formal
	definition another kind of tentacle is examined, the \emph{reduction
	tentacle}. Reduction tentacles are tentacles for which in every
	concretisation there is no outgoing edge for the identified node. This can
	be formally defined as follows:
	\begin{definition}[Reduction Tentacle]
		Let for the tentacle $(X,i)$ $v_i^{\bullet}\in V_{X^{\bullet}}$ be the 
		node such that $\con_{X^{\bullet}}(e)(i) = v_{i}^{\bullet}$ for the only
		edge $e\in E_{X^{\bullet}}$ with $\lab_{X^{\bullet}}(e) = X$ is called
		a \emph{reduction tentacle} if
		\begin{equation*}
			\forall H\in L(X^{\bullet}).\out(v_{i}^{\bullet}) = \emptyset
		\end{equation*}
	\end{definition}
	For typed nonterminals another characterisation of reduction tentacles is
	that the type of the tentacle $(X,i)$ is empty ($\type(X,i) = \emptyset$).
	As seen in \cite{LocalGreibachNormalForm} reduction tentacles can be
	determined by a syntactical analysis of the used production rules.
	Admissibility can be defined in terms of \emph{violation points} where
	violation points are tentacles that hide possibly accessible selectors.
	Formally:
	\begin{definition}[Violation Point]
		For $H\in\HC$ the tupel $(e, i)$ with $e\in E_{H}$, $\lab_{H}(e)\in N$
		and $1\leq i \leq \rk(\lab_{H}(e))$ is called a violation point, if $e$
		hides accessible selectors (i.e. there exists $e'\in E_{H}$ such that
		$\lab_{H}(e')\in\Var$ and $\con_{H}(e')(1) = \con_{H}(e)(i)$ and
		$(\lab_{H}(e),i)$ is no reduction tentacle.
	\end{definition}
	A \emph{\ac{HC}} $H$ is called \emph{admissible} if there is no violation
	point in $H$, the set of admissible \emph{\acp{HC}} is denoted by $\AHC$.
	Reconsider the abstracted representation of binary trees on page
	\pageref{tikz:abstractedbintree} which is not admissible since the tentacle
	$(B,1)$ hides the selectors $l$ and $r$ but those are actually referenceable
	by $\mathit{curr}.l$ or $\mathit{curr}.r$ respectively (when $b$ is the
	edge of the heap such that $\lab(b) = B$ then $(b,1)$ is the violation
	point). On the other hand the heap representations in Figure
	\ref{fig:bintreeconc} are admissible since the violation point is resolved
	by applying concretisation. In the following resolving violation points and
	therefore reestablishing admissibility is formalised for a \emph{\acp{HAG}}
	$G$ in the \enquote{re-admissibility} function
	\begin{equation*}
		\rea_{G}\colon\HCN\rightarrow\mathbb{P}(\AHC)
	\end{equation*}
	which maps possible abstracted \emph{\acp{HC}} to the set of admissible
	\emph{\acp{HC}} that arise by resolving violation points by applying
	production rules. But in fact not \emph{every} production rule have to be
	applied, recall therefore the \emph{\ac{fpHRG}} $G'$ of doubly linked list
	from page \pageref{eq:G'}. Imagine now a program that traverses doubly
	linked lists of arbitrary length from the last element to the first. In
	order to resolve the violation point caused by the tentacle $(L,2)$ the
	second production rule does not remove the violation point.
	But the production rule which was added to establish local
	concretisability can be used to resolve the violation point. Additionally,
	it can easily be seen that in order to establish admissibility for
	\emph{\acp{HC}} different nonterminals have to be concretised (simply
	imagine a node that is the head of a doubly linked list as well as the root
	of a binary tree and is therefore connected to a nonterminal $B$ and a
	nonterminal $L$).  Since the parts that results from concretisation of
	nonterminals are only connected via the nodes connected to the nonterminals
	and apart from that are independent it can be seen that the order in which
	\emph{\acp{HR}} are applied is insignificant to the resulting
	\emph{\ac{HG}}. This property is called \emph{confluence} and defined as
	follows:
	\begin{definition}[Confluence]
		For $H\in\HGN$ with $e_1,e_2\in E_{H}$, $e_1\neq e_2$ and $\lab_{H}(e_1),
		\lab_{H}(e_2)\in N$ it holds for two production rules
		$p_1:\lab_{H}(e_1)\rightarrow H_1$, $p_2:\lab_{H}(e_2)\rightarrow H_2$
		that
		\begin{equation*}
			H[e_1/H_1][e_2/H_2] = H[e_2/H_2][e_1/H_1]
		\end{equation*}
	\end{definition}
	As stated in \cite[p. 105]{HandbookGraphGrammars} \emph{\acp{HR}} always are
	confluent. And from this follows the following lemma (for details see
	\cite{fmsd}):
	\begin{lemma}
		\label{theo:overapprox}
		For $G\in\HAG$, $H\in\HCN$, $e\in E_{H}$ and $\lab_{H}(e)\in N$ holds
		that:
		\begin{equation*}
			L_{G}(H) = \bigcup_{\lab_{H}(e)\rightarrow K} L_{G}(H[e/K])
		\end{equation*}
	\end{lemma}
	This lemma as well as the definition of local concretisability ensures that
	applying $\rea_{G}$ still yields all possible \emph{\acp{HC}} that can be
	concretised from the \emph{\ac{HC}} $rea_{G}$ is applied to.
\subsection{Abstraction}
\label{sec:abstraction}
	After having introduced how to obtain more concrete heap
	representations from an abstracted representation in the following it is
	discussed how concrete heap representations are abstracted into more general
	representations. The idea of such an abstraction is to apply production
	rules in a backward fashion. Thus identifying the right hand side of a
	production rule, removing it and replacing it by the according nonterminal.
	Note that this approach corresponds to the way write access is handed over
	to forked processes where also subgraphs are replaced by hyperedges. For a
	comprehensible presentation of this concept recall once
	again the grammar $G$ from page \pageref{eq:G} and also consider the heap
	representation and abstraction steps in Figure \ref{fig:bintreeabs}.
	\begin{figure}
		\begin{center}
			\resizebox{\linewidth}{!}{%
				\begin{tikzpicture}[node distance=2 and 2]
	\node [node] (v1) {};
	\node [node, below left=of v1] (v2) {};
	\node [node, below right=of v1] (v3) {};
	\node [node, below left=2 and 0.5 of v3] (v4) {};
	\node [node, below right=2 and 0.5 of v3] (v5) {};
	\node [node, right =4 of v3] (v6) {};
	\node [node, above right=of v6] (v7) {};
	\node [node, below right=of v7] (v8) {};
	\node [node, below right= 2 and 0.5 of v2] (null1) {$v_{\text{null}}$};
	\node [node, below right= 2 and 0.5 of v6] (null2) {$v_{\text{null}}$};
	\node [he, label={0:$\mathit{WR}-\{\{t\}\}$}] (b) at (null2-|v8) {$B$};

	%left
	\draw [sel] (v1) to node[left]{$l_{\mathit{WR}}$} (v2);
	\draw [sel] (v1) to node[right]{$r_{\mathit{WR}}$} (v3);
	\draw [sel, bend right] (v2) to node[left]{$l_{\mathit{WR}}$} (null1);
	\draw [sel, bend left] (v2) to node[right]{$r_{\mathit{WR}}$} (null1);
	\draw [sel] (v3) to node[left]{$l_{\mathit{WR}-\{\{t\}\}}$} (v4);
	\draw [sel] (v3) to node[right]{$r_{\mathit{WR}-\{\{t\}\}}$} (v5);
	\draw [sel, bend right] (v4) to node[above]{$l_{\mathit{WR}-\{\{t\}\}}$} (null1);
	\draw [sel, bend left] (v4) to node[below]{$r_{\mathit{WR}-\{\{t\}\}}$} (null1);
	\draw [sel, bend left=55] (v5) to node[above right]{$l_{\mathit{WR}-\{\{t\}\}}$} (null1);
	\draw [sel, bend left=75] (v5) to node[below]{$r_{\mathit{WR}-\{\{t\}\}}$} (null1);

	%right
	\draw [sel] (v7) to node[left]{$l_{\mathit{WR}}$} (v6);
	\draw [sel] (v7) to node[right]{$r_{\mathit{WR}}$} (v8);
	\draw [sel, bend right] (v6) to node[left]{$l_{\mathit{WR}}$} (null2);
	\draw [sel, bend left] (v6) to node[right]{$r_{\mathit{WR}}$} (null2);
	\draw [connector] (b) to node[right]{$1$} (v8);
	\draw [connector] (b) to (null2);

	\node [right=0.6 and 0.6 of v3] (d1) {};
	\node [left=0.6 and 0.6 of v6] (d2) {};

	\draw [very thick,->, decorate,
	decoration={snake,amplitude=.7mm, segment length=4mm, post length=2mm}] (d1) to (d2);

	\begin{scope}[on background layer]
		\node [he, fit=(v3) (v4) (v5)] (layer) {};
	\end{scope}

	\draw [-, dashed, thick, color=blue!50] (layer.north east) to (b.north west); 
	\draw [-, dashed, thick, color=blue!50] (layer.south east) to (b.south west); 
\end{tikzpicture}

			}
			\caption{Heap representation of a binary tree with possible
				abstraction step}
			\label{fig:bintreeabs}
		\end{center}
	\end{figure}
	In this example the right hand side of a production rule is identified in
	the heap representation (blue mark) and replaced by the nonterminal on the
	left hand side ($B$). Note that all permissions on the right hand side of a
	production rule are the same which prevents further abstraction steps
	although besides permission the structure fits the right hand side of a
	production rule, namely the third one. But since the permissions differ
	there is no right hand side which fits this graph. For a formal approach on
	identifying the right hand side of a production rule \emph{embeddings} are
	introduced below. Embeddings are functions that identify subgraphs of
	\emph{\acp{HG}} and are formally defined as follows:
	\begin{definition}[Embedding]
		For $K,H\in\HRG$ an \emph{embedding} $\emb = (m_{V}, m_{E})$ of
		$K$ in $H$ is a pair of functions with $m_{V}:V_{K}\rightarrow V_{H}$
		and $m_{E}:E_{K}\rightarrow E_{H}$ that preserve the following
		properties:
		\begin{center}
			\begin{tabular}{|ll|}
				\hline
				$\lab_{K}(e) = \lab_{H}(m_{E}(e))$        & for all $e\in E_{K}$\\
				$\perm_{K}(e) = \perm_{H}(m_{E}(e))$      & for all $e\in E_{K}$\\
				\hline
				$m_{V}(\con_{K}(e)) = \con_{H}(m_{E}(e))$ & for all $e\in E_{K}$\\
				$\Lbag\con_{H}(e)\Rbag\cap m_{V}(V_{K}\setminus\Lbag\ext_{K}\Rbag)=
					\emptyset$                             & for all edges
					$e\in E_{H}\setminus m_E(E_{K})$\\
				\hline
				$m_{E}(e) \neq m_{E}(e')$                 & for all $e,e'\in E_{K}$
					with $e\neq e'$\\
				$m_{V}(v)\neq m_{V}(v')$                  & for all $v\in V_{K},
					v'\in V_{K} \setminus\Lbag\ext_{K}\Rbag$ with $v\neq v'$\\
				\hline
			\end{tabular}
		\end{center}
	\end{definition}
	The requirements for the functions of the embedding are as
	indicated distinguisheable in three topics:
	\begin{enumerate}
		\item Preservation, namely of labeling and permissions
		\item Structure, namely that identified edges agree on which nodes they
			connect and inner nodes are not connected to any other edge in $H$
			than those that can be also found in $K$ (since the inner nodes will
			be replaced by the nonterminal it would be unclear where those edges
			point to afterwards)
		\item Injectivity, both the embedding function of edges and nodes are
			injective with one exception: it is possible to identify multiple
			external nodes of $K$ with the same node in $H$, this causes the
			nonterminal to be connected with multiple tentacles to the same node
	\end{enumerate}
	With these definitions applying production rules backwards is defined as:
	\begin{definition}[Hyperedge introduction]
		For an \emph{\ac{HAG}} $G$, $(p_{\rho}\colon X\rightarrow M)\in G$,
		$H\in\HC$ and an embedding $(m_{V}, m_{E})$ of $K$ in $H$
		$H[M/e]\in\HGN$ is defined as follows:
		\begin{itemize}
			\item $V_{H[M/e]} = V_{H}\setminus m_{V}(V_{M}\setminus \Lbag
				\ext_{M}\Rbag)$
			\item $E_{H[M/e]} = (E_{H}\setminus m_{E}(E_{M}))\biguplus \{e\}$
			\item $\con_{H[M/e]} = \con_{H}\upharpoonright
				(E_{H}\setminus m_{E}(E_{M})) \cup \{e\mapsto m_V(\ext_{M})\}$
			\item $\lab_{H[M/e]} = \lab_{H}\upharpoonright (E_{H}
				\setminus m_{E}(E_{M})) \cup \{e\mapsto X\}$
			\item $\ext_{H[M/e]} = \ext_{H}$
			\item $\perm_{H[M/e]} = \perm_{H}\upharpoonright (E_{H}
				\setminus m_{E}(E_{M}))\cup \{e\mapsto \rho\}$
		\end{itemize}
	\end{definition}
	With these definitions an abstraction function
	\begin{equation*}
		\abs_{G}':\HGN \rightarrow\mathbb{P}(\HGN)
	\end{equation*}
	for an \emph{\ac{HAG}} $G$ is defined which maps an \emph{\ac{HG}}
	$H\in\HGN$ to the set of those \emph{\acp{HG}} that result by introducing
	successively hyperedges until there is no further embedding for any right
	hand side of any production rule in $G$. Note firstly that this way of
	abstracting terminates because $G$ is increasing and thus every abstraction
	step results in a smaller \emph{\ac{HG}} and secondly that abstraction
	might possibly lead to a set of \emph{\acp{HG}} instead of a single one as
	the following example illustrates. Consider therefore the \emph{\ac{fpHRG}}
	$G'$ from page \pageref{eq:G'} and the following example of abstraction
	steps where every arrow indicate the introduction of an hyperedge.
	\begin{center}
		\resizebox{\linewidth}{!}{%
			\begin{tikzpicture}
				\node (1) {\begin{tikzpicture}
	\node[node]              (v1) {};
	\node[node, right=of v1] (v2) {};
	\node[node, right=of v2] (v3) {};

	\draw[sel, bend left] (v1) to node[above]{$n_{\mathit{WR}}$} (v2);
	\draw[sel, bend left] (v2) to node[below]{$p_{\mathit{WR}}$} (v1);
	\draw[sel, bend left] (v2) to node[above]{$n_{\mathit{WR}}$} (v3);
	\draw[sel, bend left] (v3) to node[below]{$p_{\mathit{WR}}$} (v2);
\end{tikzpicture}

};
				\node[below left =of 1] (2) {\begin{tikzpicture}
	\node[node] (v1) {};
	\node[he, right=of v1, label=90:{$\mathit{WR}$}] (L) {$L$};
	\node[node, right=of L] (v2) {};
	\node[node, right=of v2] (v3) {};

	\draw[connector] (L) to node[above]{$1$} (v1);
	\draw[connector] (L) to node[above]{$2$} (v2);
	\draw[sel, bend left] (v2) to node[above]{$n_{\mathit{WR}}$} (v3);
	\draw[sel, bend left] (v3) to node[above]{$p_{\mathit{WR}}$} (v2);
\end{tikzpicture}
};
				\node[below right=of 1] (3) {\begin{tikzpicture}
	\node[node]              (v1) {};
	\node[node, right=of v1] (v2) {};
	\node[he, right=of v2, label=90:{$\mathit{WR}$}] (e1) {$L$};
	\node[node, right=of e1] (v3) {};

	\draw[sel, bend left] (v1) to node[above]{$n_{\mathit{WR}}$} (v2);
	\draw[sel, bend left] (v2) to node[below]{$n_{\mathit{WR}}$} (v1);
	\draw[connector] (e1) to node[above]{$1$} (v2);
	\draw[connector] (e1) to node[above]{$2$} (v3);
\end{tikzpicture}
};
				\node[below =of 2]      (4) {\begin{tikzpicture}
	\node[node]              (v1) {};
	\node[he, right=of v1, label=90:{$\mathit{WR}$}] (e1) {$L$};
	\node[node, right=of e1] (v2) {};
	\node[he, right=of v2, label=90:{$\mathit{WR}$}] (e2) {$L$};
	\node[node, right=of e2] (v3) {};

	\draw[connector] (e1) to node[above]{$1$} (v1);
	\draw[connector] (e1) to node[above]{$2$} (v2);
	\draw[connector] (e2) to node[above]{$1$} (v2);
	\draw[connector] (e2) to node[above]{$2$} (v3);
\end{tikzpicture}
};
				\node[below =of 3]      (5) {\begin{tikzpicture}
	\node[node]                                      (v1) {};
	\node[he, right=of v1, label=90:{$\mathit{WR}$}] (e)  {$L$};
	\node[node, right=of e]                          (v2) {};
	\draw[connector] (e) to node[above]{$1$} (v1);
	\draw[connector] (e) to node[above]{$2$} (v2);
\end{tikzpicture}
};
				\draw[->, thick] (1) to (2);
				\draw[->, thick] (1) to (3);
				\draw[->, thick] (2) to (4);
				\draw[->, thick, dotted] (2) to (5);
				\draw[->, thick] (3) to (4);
				\draw[->, thick] (3) to (5);
			\end{tikzpicture}
		}
	\end{center}
	Note that the dotted arrow is only valid if the third production rule (on
	page \pageref{eq:listrule3}) which established local concretisability is
	added to $G'$. Nevertheless there are two \emph{\acp{HG}} that can not be
	anymore abstracted which are obtained by introducing hyperedges repeatedly.
	As stated for \emph{\ac{HR}} it holds that two (and therefore by an
	inductive argument arbitrary many) production rules are
	applied and there is exactly one resulting \emph{\ac{HG}}, but for backward
	application there are possible multiple resulting \emph{\acp{HG}} and it
	depends on which abstraction is carried out first. Consider now adding
	additionally the production rule
	\begin{equation*}
		\label{eq:listrule4}
		\begin{tikzpicture}
	\node[he] (lhs) {$L$};
	\node[node, ext, right=1cm and 2cm of lhs] (v1) {$1$};
	\node[he, right=of v1] (L1) {$L$};
	\node[node, right=of L1] (v2) {};
	\node[he, right=of v2] (L2) {$L$};
	\node[node, ext, right=of L2] (v3) {$2$};
	
	\draw[->, very thick, shorten >=0.2cm, shorten <=0.2cm] (lhs) to (v1);
	\draw[connector] (L1) to node[above]{$1$} (v1);
	\draw[connector] (L1) to node[above]{$2$} (v2);
	\draw[connector] (L2) to node[above]{$1$} (v2);
	\draw[connector] (L2) to node[above]{$2$} (v3);
\end{tikzpicture}

	\end{equation*}
	to $G'$. This causes the above considered abstraction to result in a
	singleton set because the left hand leaf of this \enquote{abstraction tree}
	can be further abstracted to the right hand leaf. The property that
	abstraction results in a single possible \emph{\ac{HG}} is called
	\emph{backward confluence} and a \emph{\ac{HAG}} is called backward
	confluent if every abstraction for an arbitrary \emph{\ac{HG}} results in a
	single \emph{\ac{HG}}. For a backward confluent \emph{\acp{HAG}} $G$ the
	abstraction function $\abs_{G}\colon \HGN\rightarrow\HGN$ is defined as the
	mapping from every \emph{\ac{HG}} $H$ to the single element in
	$\abs_{G}'(H)$. It is unclear if it is possible to establish backward
	confluence for \emph{\acp{HAG}}, but it is decideable if an \emph{\ac{HAG}}
	is backward confluent (for details see \cite{InductivePredicates}).

	\subsection{Abstract Semantics}
	\label{sec:abstractsemantics}
	As already mentioned above and also explored in \cite{fmsd} it is possible
	to model the semantics of statements (except fork and join) on partially
	abstracted but admissible \emph{\acp{HC}} because the admissibility ensures
	that all referenceable objects and selectors are actually currently
	available in the heap representation. But it is possible that the execution
	of such a statement invalidates the admissibility because other selectors
	can become
	referenceable because a variable might get assigned a new value. This
	inadmissibility has to be resolved in order to continue the analysis of
	further statements. In order to resolve inadmissibilities the heap
	representation is at first completely abstracted and subsequently as far as
	necessary concretised to reestablish admissibility. The first abstraction
	step is used in order to minimise the heap representation because $\rea_{G}$
	stops as soon as admissibility is established.
	With these intuitions the abstract semantics (denoted by $\btr$) of pointer
	operations (which are all statements except of fork, join and assignment of
	process identifiers) can be given as follows:
	\begin{prooftree}
		\AxiomC{$(S, H)\rhd(S',I)$}
		\AxiomC{$H'\in\rea_{G}(\abs_{G}(I))$}
		\BinaryInfC{$(S,H)\btr(S',H')$}
	\end{prooftree}
	For the semantics of the assignment of process identifiers the concrete
	semantics also transfer to partially abstracted \emph{\acp{HC}} since the
	definitions given for the modifications of permissions in Section
	\ref{sec:graphtrans} are applicable for nonterminals as well, thus no
	changes are required.

	For join and fork on the other hand some modifications in semantics have to
	be introduced. At first the postponed definitions of \emph{reachability} and
	\emph{border} nodes are addressed and afterwards the semantics of fork and
	join are applied on partially abstracted \emph{\acp{HC}}.

	The set of selectors that are reachable from an initial node are those
	selectors that can be reached via moving along other selectors. For example
	consider the \emph{\ac{HC}} described in Figure \ref{fig:reachability}.
	\begin{figure}[h]
		\begin{center}
			\begin{tikzpicture}
	\node[node]                (v3) {$v_{3}$};
	\node[node, above left =of v3] (v1) {$v_{1}$};
	\node[node, below left =of v3] (v2) {$v_{2}$};
	\node[node, above right=of v3] (v4) {$v_{4}$};
	\node[node, below right=of v3] (v5) {$v_{5}$};

	\draw[sel]            (v1) to node[above, label=-90:{$e_{1}$}]{$p$} (v3);
	\draw[sel]            (v3) to node[above, label={[label distance=-0.3cm]-45:
		{$e_{2}$}}]{$p$} (v2);
	\draw[sel]            (v4) to node[above, label={[label distance=-0.3cm]-45:
		{$e_{3}$}}]{$p$} (v3);
	\draw[sel, bend left] (v2) to node[above, label=-90:{$e_{4}$}]{$p$} (v5);
	\draw[sel, bend left] (v5) to node[below, label=90:{$e_{5}$}]{$p$} (v2);
\end{tikzpicture}

		\end{center}
		\caption{simple \emph{\ac{HC}} to illustrate reachability}
		\label{fig:reachability}
	\end{figure}
	The set of reachable edges from $v_{1}$ is $\{e_{1}, e_{2}, e_{4}, e_{5}\}$
	since from $v_{1}$ $e_{1}$ is directly reachable and after moving along
	$e_{1}$ the edge $e_{2}$ is reachable and so forth. On the other hand
	$e_{3}$ is not reachable from $v_{1}$ since the selector points in the
	\enquote{wrong} direction. In order to formalise these movements along the
	edges in $H\in\HGN$ the notion of a path $\pi$ is introduced as
	\begin{equation*}
		\pi \in (\mathbb{N}\times E_{H}\times \mathbb{N})^{\ast}
	\end{equation*}
	where it holds for $\pi(k) = (i,e,j)$ and $\pi(k+1) = (i',e',j')$ that
	$\con_{H}(e)(j) = \con_{H}(e')(i')$ for all $1\leq k \leq (|\pi| - 1)$. This
	describes undirected paths, i.e.
	\begin{equation*}
		(1, e_{1}, 2)(2, e_{3}, 1)
	\end{equation*}
	is a valid path although $e_{3}$ is actually not reachable from $v_{1}$.
	Thus, not all paths agree with the intuition of reachability. The intuition
	of reachability is easy for selectors which are interpreted as directed
	edges. For nonterminals reachability is understood as the possibility to
	concretise this nonterminal in a way such that there is a reachable path
	along selectors between the connected nodes. Formally defined is
	reachability by the use of the term bridge as follows:
	\begin{definition}[Reachability]
		A bridge $(i, X, j) \in \br(N\cup\Sel)$ is called \emph{reachable}
		for $G\in\HAG$ if $i = 1, j = 2, X\in\Sel$ or if $X\in N$ and there is
		$H\in L_G(X^{\bullet})$ such that there is a reachable path $\pi$ from
		$v_i^{\bullet}$ to $v_j^{\bullet}$.
	\end{definition}
	Note that it is assumed that the set $\Sel$ is ranked by the function that
	maps all selectors to $2$ to ensure that $\br(N\cup \Sel)$ is defined. A
	path $\pi$ is called reachable if every used bridge is reachable. Let
	$\Path(u,v)$ denote all paths that start in $u$ and end in $v$, formally:
	\begin{equation*}
		\Path(u,v) = \left\{\pi\mid
			\begin{aligned}
					&\pi\text{ is a path}\\
					&\land(i,e,j) =\pi(1)\land\con_{H}(e)(i)=u\\
					&\land(i',e',j') =\pi(|\pi|)\land\con_{H}(e')(j') = v
			\end{aligned}
		\right\}
	\end{equation*}

	Furthermore it is possible to compute all reachable bridges over
	nonterminals in a \emph{\ac{HAG}} over syntactical analysis of the
	\emph{\ac{HAG}} as the following lemma states:
	\begin{lemma}
		For $G\in\HAG$ the set $\RB(G)$ denotes all \emph{reachable}
		\emph{bridges} in $G$. $\RB(G)$ can be computed by syntactical analysis
		of $G$.
	\end{lemma}
	\begin{proof}
		Let
		\begin{equation*}
			\RB_{0} =  \{(1,s,2)\mid s\in\Sel\}
		\end{equation*}	
		denote the set of all bridges over selectors in the \enquote{right}
		direction.
		Let furthermore $\NT(\pi)$ denote all used bridges in the path $\pi$ in
		an \emph{\ac{HG}} $H$
		\begin{equation*}
			\NT(\pi)=
				\left\{
					(i,\lab_H(e), j)\mid (i,e, j)\in\Lbag\pi\Rbag
				\right\}
		\end{equation*}
		then
		\begin{equation*}
			\RB_{n+1} = \{(i,X,j)\mid\exists H\in\rhs(G^{X}).
			\exists \pi\in\Path(\ext_{H}(i),\ext_{H}(j)).\NT(\pi)
			\subseteq\mathit{RB}_{n}\}
		\end{equation*}
		denote the set of all reachable bridges that make use of
		previously computed reachable bridges which ensure reachable paths. For
		example, $\RB_{1}$ is the set of bridges over selectors and those
		nonterminals for which a right hand side of a production rule exists
		where a path along selectors connects $\ext_{H}(i)$ and $\ext_{H}(j)$.
		It is easy to see that this iteration is monotone and since the set of
		all bridges for a finite set of nonterminals $N$ and selectors $\Sel$ is
		finite as well it terminates after finitly many steps. By induction over
		$n$, the depth of applied concretisation steps for nonterminals until a
		reachable path is found it follows that the fixpoint of this iteration
		is indeed $\RB(G)$.
	\end{proof}
	Combining this result with a breadth-first search that computes stepwise all
	possible paths of finite length and with the set $\RB(G)$ it can be tested
	if these paths are reachable. It follows that for a finite graph all
	reachable paths can be computed. Let $\CPath_{H}(v)$ denote all reachable
	paths in $H$ starting in the node $v$:
	\begin{corollary}
		For $H\in\HGN$, $G\in\HAG$ and $v\in V_{H}$ the set $\CPath_{H}(v)$ can
		be computed by syntactical analysis of $G$ and a structural analysis of
		$H$.
	\end{corollary}
	Another point has to be taken into account before the set of reachable edges
	can be formally defined. It is illustrated by the following production rule:
	\begin{equation*}
		\label{eq:reachex}
		\begin{tikzpicture}
	\node[he] (lhs) {$T$};
	\node[node, ext, right=1cm and 2cm of lhs] (v1) {$1$};
	\node[node, right=of v1] (v2) {};
	\node[node, ext, right=of v2] (v3) {$2$};

	\draw[->, very thick, shorten >=0.2cm, shorten <=0.2cm] (lhs) to (v1);
	\draw[sel] (v1) to node[above]{$p$} (v2);
	\draw[sel] (v3) to node[above]{$p$} (v2);
\end{tikzpicture}

	\end{equation*}
	It is evident that the bridges $(1,T,2), (2,T,1)$ are both not reachable.
	But also both tentacles $(T,1), (T,2)$ are not reduction tentacles since
	both actually abstract selectors. If any reachable path from a starting node
	$u$ to a node $v$ exists such that $v$ is connected to a $T$-labeled edge
	$e$, then $e$ has to be included in the set of reachable edges, since it
	abstracts edges which can actually be reached.
	With this in mind the set of reachable edges from a node $v$ in a
	\emph{\ac{HG}} $H$ denoted by $\reach_{\text{abs}}^{H}(v)$ can be formally
	defined in two steps. First all edges of reachable paths are gathered in
	\begin{equation*}
		P = \{e \mid (i, e, j)\in\Lbag\pi\Rbag,\pi\in\CPath_{H}(v)\}
	\end{equation*}
	and secondly all nonterminals where parts of the concretisating graphs can
	be reached in
	\begin{equation*}
		D = 
		\left\{e\mid
			\begin{aligned}
				&\lab_{H}(e)\in N, 1\leq i\leq \rk(\lab_{H}(e)).
				\exists e'\in P.\con_{H}(e)(i)\in\Lbag\con_{H}(e')\Rbag\\
				&\land (\lab_{H}(e),i)\text{ is no reduction tentacle}
			\end{aligned}
		\right\}
	\end{equation*}
	Then the set of reachable edges is defined as:
	\begin{equation*}
		\reach_{\text{abs}}^{H}(v) = P\cup D
	\end{equation*}
	and let
	\begin{equation*}
		\reach_{\text{abs}}^{H}(v_1,\dots,v_n) = \bigcup_{1\leq i\leq n}
		\reach_{\text{abs}}^{H}(v_i)
	\end{equation*}
	denote the reachable nodes from multiple initial nodes.
	This concludes computing the reachable edges in a partially abstract
	\emph{\ac{HG}}. As mentioned before the computation of reachable edges for
	fully concrete \emph{\acp{HG}} relies on the reachability in the abstracted
	case and is intuitively achieved by abstracting the concrete \emph{\ac{HG}},
	computating the set of reachable edges, concretising with the same sequence
	of production rules that is used for the abstraction to rebuild the initial
	\emph{\ac{HG}}. The set of reachable edges in the concrete
	\emph{\ac{HG}} are all edges that arise from reachable edge in the
	abstraction. Therefore \emph{production sequences}
	are introduced as sequence of tupel of production rules and edges. A
	production sequence denotes the way a heap is transformed via production
	rules. This means it is possible to revert abstraction steps by saving
	tupels of the production rules and the introduced hyperedges that arise from
	backward application of that production rule and applying this sequence in
	reversed order to the abstracted \emph{\ac{HG}} yields again the original
	concrete \emph{\ac{HG}}. Let $\pi$ be a finite production
	sequence then $\pi^{-1}$ denotes the reversed production sequence.
	Furthermore let $\pi\upharpoonright E$ denote the production sequence that
	contains all production rules in $\pi$ that are applied to edges in $E$ or
	edges that are concretised from $E$ (in possibly multiple applications of
	production rules). Let furthermore $H\xRightarrow{\pi} Q$ denote
	the application of a sequence of production rules $\pi$. Especially it holds
	for the production sequence $\pi$ which is used to obtain $\abs_{G}(H)$ that
	$\abs_{G}(H)\xRightarrow{\pi^{-1}} H$.  Additionally, let $H$ be a partially
	abstracted \emph{\ac{HG}} then let $E_{\pi}$ for an production sequence
	$\pi$ denote the set of all edges that arise from concretisation steps
	applied to edges of $E$ and edges that are originally obtained from edges in
	$E$.
	\begin{equation*}
		E_{\pi} = \{e\mid (H\upharpoonright E)\xRightarrow
			{\pi\upharpoonright E}H', e\in E_{H'}\}
	\end{equation*}
	With these notions reachability for concrete hypergraphs can be defined
	in context of a \emph{\ac{HAG}} $G$ as follows:
	\begin{equation*}
		\reach_H(v_{1},\dots,v_{n}) = (\reach_{\text{abs}}^{\abs_{G}(H)}(v_{1},
		\dots,v_{n}))_{\pi^{-1}}
	\end{equation*}
	where the abstraction preserves $v_{1}, \dots, v_{n}$ and $\pi$ denotes the
	corresponding production sequence of the abstraction.

	For the contracts of the forked programs there is one difference to the
	concrete semantics which is that a partially abstract precondition
	concretises to various possible \emph{\acp{HC}} (similar to the contracts
	presented in \cite{ProcedureSummaries}). Just like concretisation
	of nonterminals demands that all production rules that establish
	admissibility are analysed (since the information from which of these
	concrete \emph{\acp{HC}} it is abstracted from is lost) there are also
	multiple postconditions that might apply after the execution of a program
	depending on the actual structures the partially abstract precondition
	concretises to. Formally, this leads to a set of contracts for a program $m$
	where one precondition leads to multiple postconditions:
	\begin{equation*}
		\Cont_{\text{abs}}(m)\subseteq
		\mathbb{P}((\HGN\times\mathbb{P}(E)\times
		\mathbb{P}(\mathit{HG}^{\PI'}_{\Sigma_{N}'})))
	\end{equation*}
	where for every $C = (P_{C}, E_{C}, \mathbb{Q}_{C})\in\Cont_{\text{abs}}(m)$
	holds that $E_{C}\subseteq E_{\mathit{WR}}^{P_{C}}$.
	Furthermore there is for every program $m$ and
	precondition $P$ maximal one contract $C\in\Cont_{\text{abs}}(m)$. Because
	if there are two contracts $C,C'\in\Cont_{\text{abs}}(m)$ with
	$P_C\afiso P\afiso P_{C'}$ then those contracts can be united to a contract
	$C'' = (P, E_{C}\cup E_{C'}, \mathbb{Q}')$ such that $\mathbb{Q}'$ contains
	all possible results from executions of $m$ starting from the initial
	heap state (which is obtained from the precondition and the set of
	alternable edges the same way it is described in Section \ref{sec:join}
	with straightforward ajustments for nonterminals).

	As already introduced the intuition for border nodes is that these nodes
	are part of the subgraph for which the access tickets are transferred to the
	newly forked process but are also still part and accessible (with limitation
	to some selectors) to the forking process.
	For the formal definition let $H$ be the current heap representation, 
	$C = (P_C, E_C, \mathbb{Q}_C)$ be a contract for the forked process, $R$ be
	the reachable subgraph from the actual parameter of the fork statement and
	it holds that $R\afiso P_C$. Then the set of border nodes can be defined as
	follows:
	\begin{equation*}
		\border_{\text{abs}}^{H}(C) = \left(\bigcup_{e\in E_{C}}\Lbag\con_{H}(e)
		\Rbag\right)\bigcap\left(\bigcup_{e\in E_{H}\setminus E_{C}}
		\Lbag\con_{H}(e)\Rbag\right)
	\end{equation*}
	which leads to the following transformation of the heap representation of
	the forking process:
	\begin{equation*}
		H' = H[\downarrow t][\setminus E_{C}][+_\mathit{WR}N_{\{t\}}
		\rightrightarrows \mathit{enum}_{b_{C}(H)}]
		[(E_{P_{C}}\setminus E_{C}) - \{t\}]
	\end{equation*}
	where $t$ denotes the process identifier that identifies the newly forked
	process initially.
	With these definitions the abstract semantics for the fork statement can be
	given as follows where
	$R=(H\upharpoonright \reach_{\text{abs}}(\llbracket x_{1}\rrbracket_{H},
		\dots,\llbracket x_{n}\rrbracket_{H}))$
	\begin{prooftree}
		\AxiomC{$(P_{C},E_{C},\mathbb{Q}_{C})\in\Cont_{\text{abs}}(m)$}
		\AxiomC{$P_{C} \afiso R$}
		\AxiomC{$H'' \in \rea_{G}(\abs_{G}(H'))$}
		\TrinaryInfC{$(t=\text{fork}(m(x_{1},\dots,x_n)),H)\btr(\varepsilon,H'')
		$}
	\end{prooftree}

	Lastly, the join statement is dealt with almost the same as in the
	concrete semantics. Recall therefore the two transformations steps for one
	precondition $Q$ from page \pageref{eq:joinH}:
	\begin{equation*}
		Q'_{1}=Q_{C}[\downarrow\mathit{enum}_{\mathit{Var}'_{\text{thread}}}(1)]
		\dots[\downarrow\mathit{Var}'_{\text{thread}}
		(|\mathit{Var}'_{\text{thread}}|)]
	\end{equation*}
	and
	\begin{equation*}
		Q'_{2} = Q'_{1}[\setminus (E^{Q'_{1}}_{\mathit{RD}}\cup
			E^{Q'_{1}}_{\mathit{RD^{\ast}}})]
	\end{equation*}
	then there are again two transition rules for the join statement where
	one deals with fully returned read tickets and the other one with lost
	read tickets. Especially \emph{every} postcondition $Q\in\mathbb{Q_C}$
	is examined (since every postcondition describes a valid execution of
	the program from the initial heap state).
	\begin{prooftree}
		\AxiomC{$Q\in\mathbb{Q}_{C}$}
		\AxiomC{$E^{Q'_{1}}_{\mathit{RD^{\ast}}}\neq\emptyset$}
		\AxiomC{$H[\downarrow T_{t}]\xRightarrow{N_{T_{t}}\rightarrow Q'_{2}}H'$}
		\AxiomC{$H''\in\rea_{G}(\abs_{G}(H'))$}
		\QuaternaryInfC{$(\text{join}(t), H)\btr(\varepsilon,H'')$}
	\end{prooftree}
	\begin{prooftree}
		\AxiomC{$Q\in\mathbb{Q}_{C}$}
		\AxiomC{$E^{Q'_{1}}_{\mathit{RD^{\ast}}}=\emptyset$}
		\AxiomC{$H[\leftarrow T_{t}]\xRightarrow{N_{T_{t}}\rightarrow Q'_{2}}H'$}
		\AxiomC{$H''\in\rea_{G}(\abs_{G}(H'))$}
		\QuaternaryInfC{$(\text{join}(t), H)\btr(\varepsilon,H'')$}
	\end{prooftree}
	where $C = (P_{C}, E_{C}, \mathbb{Q}_{C})$ is the contract which the process
	is forked by before (which can be obtained by attaching it to the
	placeholder when the fork statement is executed). Additionally it can be
	explained in the context of abstraction why it is avoided for the concrete
	semantics as well as for the abstract semantics to match the shared edges
	from the postcondition to the ones in the heap representation of the joining
	process. Because of the applied abstraction it is possible that nonterminals
	that are identified at the join statement get concretised and abstracted
	differently through the execution of both processes. To avoid accounting
	from which nonterminals different edges arose (especially since it had to be
	done for every forked process individually) it is just dealt with by
	handing over one read ticket for all shared edges and returning the minimum
	of this read ticket (nothing if any edge cannot return its ticket completely
	or everything if every edge can guarantee to return the whole ticket). Also
	this fits the approach to deal with abstraction and permissions as
	orthogonal concepts (like it is approached by introducing fully permissive
	grammars on page \pageref{def:fpg}).
	Note that again one abstraction and one
	concretisation step is applied to the resulting heap representation. This is
	done because after the transformation of permissions as well as inserting
	the \enquote{$\mathit{WR}$ part} into the heap representation it is possible
	that the right hand side of production rule can be found in the resulting
	heap representation which were not present before. Thus, to obtain the
	most general heap representation the abstraction step is executed and
	following to obtain the minimal admissible \emph{\acp{HC}} one
	concretisation step is applied.

	\subsection{Correctness}
	The main result justifying the taken approach on the presented abstraction
	is to show that $\btr$ is an over-approximation of the transition relation
	$\rhd$. Therefore it is assumed that abstraction and concretisation relies
	on a backward confluent \emph{\ac{HAG}} $G$ and abstract and concrete
	contracts are connected in the following way: For every
	$C = (P_{C}, E_{C}, Q_{C})\in\Cont(m)$ exists
	$C' = (P_{C'}, E_{C'}, \mathbb{Q}_{C'})\in\Cont_{\text{abs}}(m)$ such that
	\begin{enumerate}[label=(\roman*)]
		\item $P_{C}\in L_{G}(P_{C'})$, which implies there is a production
			sequence $\pi$ such that $P_{C'}\xRightarrow{\pi}P_{C}$
		\item  $P_{C'}\upharpoonright E_{C'}
			\xRightarrow{\pi\upharpoonright E_{C'}}P_{C}\upharpoonright E_{C}$
		\item there is $Q'\in\mathbb{Q}_{C'}$ with $Q_{C}\in L_{G}(Q')$
	\end{enumerate}
	This intuitively means that
	\begin{enumerate*}[label=(\roman*)]
		\item preconditions of concrete contracts arise from concretisation of 
			preconditions of abstract contracts, where
		\item every edge in the alternable set of the concrete contract arise
			from concretisation of edges in the alternable set of the abstract
			contract and
		\item the postcondition of the concrete contract can be found by
			concretising one of the postconditions of the abstract contract.
	\end{enumerate*}
	Note further that the order of $\pi$ does not matter due to the confluence
	property of \emph{\ac{HR}}, but a previous \emph{\ac{HR}} can expose the
	hyperedges which later production rules are applied to. This might cause
	dependencies between the production rules. A subsequence $\chi$ of
	$\pi$ denoted as $\chi\prec\pi$ is a sequence of production rules such that
	these production rules appear also in $\pi$.

	These assumptions are essential requirements for the presented analysis and
	abstraction approach of this paper. Because the actual analysis is actually
	designed for abstracted representation, it is expected that the set of
	abstracted contracts is computed first and the concrete contracts are simply
	generated by concretisation of abstracted contracts. This justifies that
	even for the concrete semantics a corresponding abstracted contract exists
	for every concrete contract. This is necessary because the border nodes for
	the concrete semantics are computed by abstracting and computing the border
	nodes for the abstracted case and concretise by the sequence of production
	rules used for the abstraction. This is necessary to ensure that the
	abstract semantics overapproximate the concrete semantics. This approach
	agrees with the intuition of the border
	nodes in the concrete semantics because the computation of border nodes in
	the abstracted \emph{\ac{HC}} is a superset of the actual border nodes.
	Recall therefore the intutition of border nodes for the
	concrete semantics as all nodes that are connected to an edge for which
	$\mathit{WR}$ permission is transferred and an edge that is still present 
	within the heap representation of the forking process. Let $H$ be this heap
	representation and $C = (P_{C}, E_{C}, Q_{C})$ the contract by which the
	fork statement is executed then the intuition translates to the following
	set:
	\begin{equation*}
		\mathit{IB}_{H}(C) = \{v\mid \exists e\in E_{C}.\exists e'\in E_{H}
			\setminus E_{C}.v\in\Lbag\con_{H}(e)\Rbag
			\cap\Lbag\con_{H}(e')\Rbag\}
	\end{equation*}
	Then the following lemma states that this computation
	ensures that the border nodes are a superset of the nodes that are
	intuitively understood as border nodes:
	\begin{lemma}[Border Lemma]
		For $C\in\Cont(m)$ and $C'\in\Cont_{\text{abs}}(m)$ such that for $C$ and
		$C'$ $(i),(ii),(iii)$ holds that
		$\mathit{IB}_{H}(C)\subseteq\border_{H}(C')$
	\end{lemma}
	Firstly one additional property for
	\emph{\ac{HR}}, namely \emph{context-freeness}, is presented in
	the following because it motivates some of the used results for the
	presented proofs. But in order to avoid some formal machinery
	context-freeness is only presented informally, for formal details as well as
	the proof that \emph{\ac{HR}} actually is context-free see
	\cite[pp. 111-115]{HandbookGraphGrammars}: The deriveration of nonterminals
	is context-free in the sense that it is independent from the rest of the
	\emph{\ac{HG}}. This means, first applying production rules to nonterminals
	and glueing the resulting hypergraphs together yields the same result as
	glueing the hypergraphs together and applying the same production rules
	afterwards. In the following the proof for the \emph{Border Lemma} is
	presented which actually makes use of context-freeness property:
	\begin{proof}[Border Lemma]
		Let $v\in\mathit{IB}_{H}(C)$ be arbitrarily chosen. It follows that
		$v\in\border_{H}(C)$ by the following argument:
		Since $v\in\mathit{IB}_{H}(C)$ it follows that there is $e\in E_{C}$ and
		$e'\in E_{H}\setminus E_{C}$ such that
		$v\in\Lbag\con_{H}(e)\Rbag\cap\Lbag\con_{H}(e')\Rbag$. Let furthermore
		$C'$ denote the abstract contract to $C$ for which $(i), (ii), (iii)$
		holds. By the context-freeness property of \emph{\ac{HR}} and because
		$(ii)$ holds follows that
		there are $e_{\text{abs}}\in E_{C'}$ and $e'_{\text{abs}}\in
		E_{\abs_{G}(H)}\setminus E_{C'}$ such that $e$ arises from concretisation
		of $e_{\text{abs}}$ and $e'$ from concretisation of $e'_{\text{abs}}$.
		Therefore it follows immediatly that
		$v\in\Lbag\con_{\abs_G(H)}(e_{\text{abs}})\Rbag\cap
		\Lbag\con_{\abs_G(H)}(e'_{\text{abs}})\Rbag$ and hence
		$v\in\border_{C'}(H)$.
	\end{proof}
	This proves that it is viable to rely on abstraction in order to determine
	the border nodes because it overapproximates the definition of border nodes
	for the concrete case.

	In order to proof the overapproximation of $\btr$ it is actually shown that
	if $(S, H)\rhd(S', H')$ that there is $K$ such that
	$(S,\abs_{G}(H))\btr(S',K)$ and $H'\in L_{G}(K)$. Furthermore the proof is
	only presented for the fork and join statement and the assignment of process
	identifier. For all other cases the overapproximation can be shown
	straigthforwardly by adapting the proof of overapproximation presented in
	\cite{fmsd}. Especially noteworthy in this context is that the
	presented $\abs_{G}$ and $\rea_{G}$ functions of this paper with Theorem
	\ref{theo:overapprox} and that every backwards application of a production
	rule can be undone by forwards application of the same production rule
	satisfy the requirements for the concretisation and abstraction functions
	demanded by the \emph{Correctness Theorem} in \cite[p. 19]{fmsd}.

	For the following proofs of overapproximation for fork and join some general
	arguments are presented in front to reduce the formal complexity of the
	actual argumentation:
	\begin{enumerate}[label=(\Roman*)]
		\item \label{arg:strictpropagation}
			Because permissions propagate
			strictly through production rules it follows for a production sequence
			$\pi$ and two \emph{\acp{HG}} $H,Q$ with $H\xRightarrow{\pi} Q$ and a
			permission $\rho$ that
			\begin{equation*}
				(E^H_{\rho})_{\pi} = E^{Q}_{\rho}
			\end{equation*}
			because every nonterminal with permission $\rho$ can only concretise
			to edges with permission $\rho$ and also edges with permission $\rho$
			can only be abstracted into nonterminals with permission $\rho$.
		\item \label{arg:preservation}
			Since production rules are generally
			assumed to not abstract placeholder and variables it can be assured
			that nodes that are identified by variables are preserved by
			abstractions and introduced placeholder are preserved through
			concretisation.
	\end{enumerate}

	\begin{lemma}[Overapproximation of Fork]
		For a backward confluent $G\in\HAG$, $H,H'\in\HC$ and
		$(t=\text{\textbf{fork}}(m(x_{1},\dots,x_{n})),H)\rhd(\varepsilon, H')$
		it holds that there is $I\in\HCN$ with
		$(t=\text{\textbf{fork}}(m(x_1,\dots,x_n)),\abs_{G}(H))
			\btr(\varepsilon,I)$ and $H'\in L_{G}(I)$.
	\end{lemma}
	\begin{proof}
		Let $C$ be the contracted by which
		$(t=\text{\textbf{fork}}(m(x_{1},\dots,x_{n})),H)\rhd(\varepsilon, H')$
		is computed. Then by the definition of $\rhd$ it follows that
		$H' = H[\downarrow t][\setminus E_{C}]
		[+_{\mathit{WR}}N_{\{t\}}\rightrightarrows
		\enum_{b_{C}(H)}][(E_{P}\setminus E_{C})-\{t\}]$. Let $C'$ be the
		abstract contract such that $C$ and $C'$ satisfy $(i), (ii), (iii)$.
		With this it is shown that $H'\in L_{G}(\abs_{G}(H)[\downarrow t]
		[\setminus E_{C'}][+_\mathit{WR}N_{\{t\}}
		\rightrightarrows \mathit{enum}_{b_{C'}(\abs_{G}(H))}]
		[(E_{P_{C'}}\setminus E_{C'}) - \{t\}])$ by the following argument:
		Obviously there is a production sequence $\pi$ such that
		$\abs_{G}(H)\xRightarrow{\pi}H$. Furthermore the permissions in $H$ and
		in $\abs_{G}(H)$ are altered both by $[\downarrow t]$ the same way, also
		because placeholders cannot be abstracted (by argument (II)) and because
		the permission propagate through concretisation (by argument (I)) it
		follows that there is a production sequence $\pi'$ that mirrors $\pi$ but
		adapts the permissions of production rules according to the edges these
		production rules are applied to such that $\abs_{G}(H)[\downarrow t]
		\xRightarrow{\pi'}H[\downarrow t]$. Secondly, it can be assured by the
		definition of
		$\reach_{H}$ and because $\llbracket x_1\rrbracket_H,\dots,
		\llbracket x_n\rrbracket_H$ are preserved through abstraction
		(by argument (II)) that for $R\coloneqq \reach_{\text{abs}}^{\abs_{G}(H)}
		(\llbracket x_1
		\rrbracket_{\abs_{G}(H)}, \dots,\llbracket x_n\rrbracket_{\abs_G(H)})$
		it follows that $\underbrace{\abs_{G}(H)\upharpoonright R}_{P_{C'}}
		\xRightarrow{\pi'\upharpoonright R} \underbrace{H\upharpoonright
		\reach_{H}(\llbracket x_1\rrbracket_H,\dots,\llbracket x_n\rrbracket_H)}_
		{P_C}$. Because of the context-freeness of \emph{\ac{HR}} and property
		$(ii)$ for $C$ and $C'$ it follows that removing $E_{C'}$ in
		$\abs_{G}(H)[\downarrow t]$ removes $E_{C}$ in $H[\downarrow t]$.
		This implies that there is a production sequence $\pi''$ which is the
		same as $\pi'$ but reduced to those production rules for which the edges
		are actually present after removing $E_{C'}$ in $\abs_{G}(H)$ such that
		$\abs_{G}(H)[\downarrow t][\setminus E_{C'}]\xRightarrow{\pi''}
		H[\downarrow t][\setminus E_{C}]$. Since the border nodes in both cases
		are determined the same way and (II) holds it follows immediatly that
		$\abs_{G}(H)[\downarrow t][\setminus E_{C'}]
		[+_{\mathit{WR}}N_{\{t\}}\rightrightarrows
		\enum_{b_{C'}(\abs_{G}(H))}]\xRightarrow{\pi''}
		H[\downarrow t][\setminus E_{C}][+_{\mathit{WR}}N_{\{t\}}
		\rightrightarrows\enum_{b_{C}(H)}]$. Finally, because of (I) it follows
		that there is $\pi'''$ such that
		$\abs_{G}(H)[\downarrow t]
		[\setminus E_{C'}][+_\mathit{WR}N_{\{t\}}
		\rightrightarrows \mathit{enum}_{b_{C'}(\abs_{G}(H))}]
		[(E_{P_{C'}}\setminus E_{C'}) - \{t\}]
		\xRightarrow{\pi'''}H[\downarrow t][\setminus E_{C}]
		[+_{\mathit{WR}}N_{\{t\}}\rightrightarrows
		\enum_{b_{C}(H)}][(E_{P}\setminus E_{C})-\{t\}]$ where $\pi'''$ mirrors
		$\pi''$ but adapts the permissions of the production rules according to
		the permissions of the edges they are applied to. This concludes that
		$H'\in L_{G}(\abs_{G}(H)[\downarrow t]
		[\setminus E_{C'}][+_\mathit{WR}N_{\{t\}}
		\rightrightarrows \mathit{enum}_{b_{C'}(\abs_{G}(H))}]
		[(E_{P_{C'}}\setminus E_{C'}) - \{t\}])$ and because $\rea_{G}$ preserves
		the language of the \emph{\ac{HG}} it is applied to it follows that there
		is $I\in\HCN$ such that $(t=\text{\textbf{fork}}(m(x_1,\dots,x_n)),
		\abs_{G}(H))\btr(\varepsilon, I)$ with $H'\in L_{G}(I)$.
	\end{proof}

	And secondly the join statement is examined in detail as follows:
	\begin{lemma}[Overapproximation of Join]
		For a backward confluent $G\in\HAG$, $H,H'\in\HC$ with
		$(\text{\textbf{join}}(t),H)\rhd(\varepsilon, H')$
		it holds that there is $I\in\HCN$ such that
		$(\text{\textbf{join}}(t),\abs_{G}(H))
			\btr(\varepsilon,I)$ and $H'\in L_{G}(I)$.
	\end{lemma}
	\begin{proof}
		Let $T_t$ denote the token of identifiers that identify the process $t$
		and $C=(P_{C}, E_{C}, Q_{C})\in\Cont(m)$ the contract by which the
		process identified by all $t'\in T_t$ is joined. Let further more denote
		$C'=(P_{C'}, E_{C'}, \mathbb{Q}_{C'})\in\Cont_{\text{abs}}(m)$ such that
		$C$ and $C'$ satisfy $(i),(ii),(iii)$. Therefore there is (at least one)
		$Q_{C'}\in\mathbb{Q}_{C'}$ such that $Q_{C}\in L_{G}(Q_{C'})$. This
		implies there is a production sequence $\pi$ such that
		$Q_{C'}\xRightarrow{\pi}Q_{C}$. Furthermore it is clear that there is a
		production sequence $\lambda$ with $\abs_{G}(H)\xRightarrow{\lambda}H$.
		Additionally it is $Q_{1}=Q_{C}[\downarrow\enum_{\PI'}(1)]\dots
		[\downarrow\enum_{\PI'}(|\PI'|)]$ and $Q_{2}=Q_{1}[\setminus
		(E^{Q^{1}}_{\mathit{RD}}\cup E^{Q_{1}}_{\mathit{RD}^{\ast}})]$, and
		accordingly $Q_{1}^{\text{abs}}=Q_{C'}[\downarrow\enum_{\PI'}(1)]\dots
		[\downarrow\enum_{\PI'}(|\PI'|)]$ and
		$Q_{2}^{\text{abs}}=Q_{1}^{\text{abs}}
		[\setminus(E^{Q_{2}^{\text{abs}}}_{\mathit{RD}}\cup
		E^{Q_{2}^{\text{abs}}}_{\mathit{RD^{\ast}}})]$. By argument (I) and
		because the successively dropped process identifier alternate the
		permissions in $Q_{C}$ and $Q_{C'}$ the same way (an analogous case is
		examined in the proof of the overapproximation for the fork statement
		above) it can be savely assumed that
		$Q_{1}^{\text{abs}}\xRightarrow{\pi'}Q_{1}$ where $\pi'$ mirrors $\pi$
		but adapts the permissions of the production rules to fit the permissions
		of the edges they are adapted to, and furthermore it follows that
		$E^{Q_{1}}_{\mathit{RD}^{\ast}} = \emptyset$ if and
		only if $E^{Q_{1}^{\text{abs}}}_{\mathit{RD}^{\ast}}=\emptyset$, since
		every $e$ with $\mathit{RD}^{\ast}$ in $Q_{1}$ has to arise from an
		edge with $\mathit{RD}^{\ast}$ permission in $Q_{1}^{\text{abs}}$ and
		every edge in $Q_{1}^{\text{abs}}$ with an $\mathit{RD}^{\ast}$
		permission concretises to edges with $\mathit{RD}^{\ast}$ permissions.
		This implies that the abstract semantics as well as the concrete
		semantics agree upon which of both production rules is used in both
		cases.
		Additionally because the \enquote{write-part} and the \enquote{read-part}
		of the postcondition can be distinguished by their $\mathit{BasePerm}$
		and of argument (I) there is a production sequence $\pi''$ which is
		the same as $\pi'$ restricted to the edges with
		$\mathit{WR},\mathit{WR}^{\ast}$ permissions ($\pi'' =\pi'\upharpoonright
		E^{Q_{1}^{\text{abs}}}_{\mathit{WR}}\cup
		E^{Q_{1}^{\text{abs}}}_{\mathit{WR}^{\ast}}$) such that
		$Q_{2}^{\text{abs}}\xRightarrow{\pi''}Q_{2}$. Let now in the following
		$\Delta\in\{\downarrow T_t,\leftarrow T_t\}$ denote the graph
		transformation which is applied to the heap representation to return the
		\enquote{read-part} of the postcondition. It is already established that
		$\abs_{G}(H)\xRightarrow{\lambda}H$. Because both transformations change
		the permissions in $\abs_{G}(H)$ and $H$ the same way, this implies that
		there is $\lambda'$ which mirrors $\lambda$ aside the permissions which
		are adapted to fit the edges the production rules are applied to such
		that $\abs_{G}(H)[\Delta]\xRightarrow{\lambda'}H[\Delta]$. Let in the
		following $K\in\HCN$ denote the \emph{\ac{HC}} such that
		$\abs_{G}[\Delta] \xRightarrow{N_{T_t}\rightarrow Q_{2}^{\text{abs}}} K$
		holds. And $H'$ is by the definition of the production rules for the
		concrete semantics (see page \pageref{prooftree:join}) the \emph{\ac{HC}}
		such that $H[\Delta]\xRightarrow{N_{T_t}\rightarrow Q_{2}} H'$. It
		follows from the context-freeness and independence of $\pi''$ and
		$\lambda'$ (i.e. no production rule in $\pi''$ is needed to reveal edges
		which production rules of $\lambda$ are applied to and vice versa) that
		$K\xRightarrow{\lambda'} \underbrace{K'}_{\text{intermediate state}}
			\xRightarrow{\pi''} H'$, where the intermediate step $K'$ is fully
		concrete in the part around the inserted $Q_{2}^{\text{abs}}$ which is
		then concretised by $\pi''$.
		Therefore
		$K\xRightarrow{\lambda'\pi''}H'$ which implies $H'\in L_{G}(K)$.
		Finally, because application of abstraction and concretisation yields at
		least the language of the \emph{\ac{HC}} they are applied to it follows
		that there is $I\in\rea_{G}(\abs_{G}(K))$ such that $H'\in L_{G}(I)$.
	\end{proof}
	At last the overapproximation of the assignment of process identifier is
	given.
	\begin{lemma}[Overapproximation of Assignment of Process Identifier]
		For a backward confluent $G\in\HAG$, $H,H'\in\HC$ with
		$(t = t',H)\rhd(\varepsilon, H')$
		it holds that there is $I\in\HCN$ such that
		$(t = t',\abs_{G}(H))
			\btr(\varepsilon,I)$ and $H'\in L_{G}(I)$.
	\end{lemma}
	\begin{proof}
		This follows immediatly from argument (I) and because $[t = t']$ operates
		on the permissions of the concrete \emph{\ac{HC}} as well as the
		corresponding abstraction the same way.
	\end{proof}
	This concludes the proof that $\btr$ is an overapproximation of $\rhd$.
	$\blacksquare$
